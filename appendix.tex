\appendix
\chapter*{Appendices}
\addcontentsline{toc}{chapter}{Appendices}
\renewcommand{\thesection}{\Alph{section}}

\section{System instructions}
In order to run the core system the following dependencies are required:
\begin{itemize}
\item Python 2.7
\item Theano
\item Numpy
\item Unirest
\end{itemize}

The system have only been tested on Ubuntu 14.04, but should be able to run on both Windows and Linux as long as the listed dependencies have been installed. Ubuntu is highly recommended because of a more convenient installation process. \\

A Nvidia GPU is also highly recommended for running the system. In most instances a GPU can give considerable speed improvements when training compared to a CPU. This is critical when having to deal with large datasets and models with millions of parameters. In order for Theano to efficiently use your GPU while training, CUDA Toolkit has to be installed. \\

A complete installation guide for Ubuntu can be found in Appendix \ref{app:ubuntuInstall}


A graphical user interface can also be utilized for monitoring training. Running experiments can also be stopped from this user interface. In addition all experiment data is stored as JSON, and can be viewed in the user interface. This includes, loss curve, precision recall curve and hyperparameters used while training. \\

The monitoring system can either be run locally, or installed on a server. All communication between the core system and the monitoring system is done by http. To enable monitoring, enable\_gui should be set to true in the core system's config file.

Utilizing this monitoring system requires these dependencies:
\begin{itemize}
\item Node.js
\item MongoDB
\end{itemize}

For an installation instructions, see Appendix \ref{app:monitorInstall}.


\section{Ubuntu Installation Guide}
\label{app:ubuntuInstall}
This guide will install all dependencies required for running Theano with a GPU. This guide can also be found in the repository's README file (https://github.com/olavvatne/CNN).
\\

\noindent Install all dependencies:
\begin{lstlisting}[language=bash]
  $ sudo apt-get install -y gcc g++ gfortran build-essential git
   wget linux-image-generic libopenblas-dev python-dev python-pip 
   python-nose python-numpy python-scipy  
\end{lstlisting}


\noindent Install Theano:
\begin{lstlisting}[language=bash]
  $ sudo pip install --upgrade --no-deps 
  git+git://github.com/Theano/Theano.git
\end{lstlisting}


\noindent Download Cuda 7 toolkit:
\begin{lstlisting}[language=bash]
  $ sudo wget 
  http://developer.download.nvidia.com/compute/cuda/repos/ubuntu1404/x86_64/cuda-repo-ubuntu1404_7.0-28_amd64.deb
\end{lstlisting}


\noindent Depackage Cuda:
\begin{lstlisting}[language=bash]
  $ sudo dpkg -i cuda-repo-ubuntu1404_7.0-28_amd64.deb  
\end{lstlisting}


\noindent Install the cuda driver:
\begin{lstlisting}[language=bash]
  $ sudo apt-get update
  $ sudo apt-get install -y cuda  
\end{lstlisting}

\noindent Update the path to include cuda nvcc and ld\_library\_path, and then reboot:
\begin{lstlisting}[language=bash]
  $ echo -e "
  export PATH=/usr/local/cuda/bin:$PATH
  export LD_LIBRARY_PATH=/usr/local/cuda/lib64" >> .bashrc  
\end{lstlisting}

\noindent  Create a theano config file. Name this file .theanorc and place it in your home directory:\todo{FIX file}\\
\begin{tabular}{ l }
  \hline			
  ~[global] \\
  floatX=float32 \\
  device=gpu \\
  mode=FAST\_RUN \\
  \\
  ~[nvcc] \\
  fastmath=True \\
  \\
  ~[cuda] \\
  root=/usr/local/cuda \\
  \hline  
\end{tabular}

\section{Monitoring Interface Installation Guide}
\label{app:monitorInstall}