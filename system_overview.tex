\section{System Overview}
\label{sec:systemOverview}
\todo[inline]{Super overview.
Dataset loader (Pre generated or otherwise) Convolutional neural network. Storage. webGUI and in file
}
\todo[inline]{In big terms, talk about finding roads, but as mnih. Datasets contain two label noise. Investigate ways of mitigating the impact label noise have. Also the aerial image datasets contain large variety of images. Some are much harder. Epecially when inconsistent labelling. Curriculum learning. }

\todo[inline]{Too many details presented here. This should be much shorter and give a birds view of the entire system }

\todo[inline]{dataloader short intro}
\todo[inline]{Data loader and network loosely coupled. Can adapt system to other image problems easily. All hyperparameters can be changed in a config file. From number of layers, epochs etc etc. What backpropagation method}

\todo[inline]{Short intro to CNN}
\todo[inline]{Reasons to use CNN. Related work , compelling results. CNN extract features. Suited for image task. }



The road extraction system was written in Python, and uses the open source library Theano. Theano enables the user to define and evaluate mathematical expressions involving tensors. The library implements several useful features for developing \ac{CNN}s, such as back-propagation, convolution, and max pooling. Training deep neural networks on a \ac{GPU} can be considerably faster than on a \ac{CPU}. Theano can utilize both the \ac{CPU} and \ac{GPU} without making any modifications to the code.\\





\todo[inline]{Mention the webgui. Store results and display stuff}