\section{Structured Literature Review}
The purpose of conducting a \ac{SLR} is to get an overview of the field of remote image sensing, as well as the research related to curriculum learning and dealing with noisy labels. The \ac{SLR} method has been chosen to investigate these topics, and provides a formal way of identifying the information available.

\subsection{Identification of research}
This section outlines the strategy that was utilized to search for primary studies. By utilizing a search strategy, literature relevant to the defined research questions can be identified and collected. Search terms have been defined, as well as literature resources.

\subsubsection{Literature resources}
\label{sec:literature_resources}
The following resources was searched in order to identify and collect relevant material:
\begin{itemize}
	\item ACM digital library
	\item IEEExplore
	\item ScienceDirect
	\item CiteSeer
	\item Springer Link
	\item ECCV (conference)
	\item NIPS (conference)
\end{itemize}

\subsubsection{Key terms and groupings}
Table \ref{tab:search_terms} describes the different word groups employed by the search strategy. Each group specifies a number of terms that are either synonyms or related to each other. Additional terms have been appended, as a result of a search strategy validation.


\begin{table}[htp]

\caption{The terms and groups}
\begin{center}
\begin{adjustbox}{max width=\textwidth}
\begin{tabular}{|c|c|c|c|c|c|c|c}\hline
 		& Group 1 & Group 2 & Group 3 & Group 4 & Group 5 & Group 6\\\hline
Term 1 	& \pbox{20cm}{Aerial \\ images} & \pbox{2cm}{Curriculum learning} & \pbox{2cm}{Noisy \\ labels} & \pbox{2cm}{Neural \\ network} & Segmentation & Roads\\\hline
Term 2 	& \pbox{20cm}{Satellite \\ images} & \pbox{2cm}{Guided learning} & \pbox{2cm}{Missing labels} & \pbox{3cm}{Convolutional \\ neural network} & Classification & \\\hline
Term 3 	& \pbox{20cm}{Remote \\ sensing} & \pbox{2cm}{Example ordering} & \pbox{2cm}{Semi-supervised} & \pbox{3cm}{Machine learning} & & \\\hline
Term 4 	& \pbox{20cm}{Images} & & \pbox{2cm}{Noisy data \\ (appended)} & \pbox{3cm}{Deep neural \\ networks \\ (appended)} & & \\\hline
\end{tabular}
\end{adjustbox}
\end{center}
\label{tab:search_terms}
\end{table}

\subsubsection{Search strategy}
Based on Table \ref{tab:search_terms} several search expressions were devised. Each of these expressions were run for each of the resources listed in the literature resources section. 

\begin{itemize}
	\item (aerial images OR satellite images) AND (segmentation OR classification)
	\item (remote sensing OR aerial images) AND (noisy labels OR missing labels OR semi-supervised OR noisy data)
	\item (neural network OR machine learning OR convolutional neural networks) AND (aerial images OR satellite images)
	\item (curriculum learning OR guided learning) AND (machine learning OR neural networks OR deep neural networks OR  convolutional neural network) 
	\item (segmentation OR classification) AND (roads)
	\item (Noisy labels OR missing labels OR semi-supervised OR noisy data) AND (machine learning OR neural network OR convolutional neural network)
	\item (aerial) AND (noisy labels OR missing labels OR semi-supervised OR noisy data) AND roads
\end{itemize}

\subsection{Selection process}
The search expressions resulted in a large number of hits for each resource. The top 15 results for each expression were stored, and a  selection process was created to reduce the over 400 studies to a more manageable number. At first, the title of each study was evaluated. If the title seemed unrelated to the research goal or the research questions found in Section \ref{sec:Goals and Research Questions}
the study was removed. Then the title and the abstract were evaluated using the inclusion criteria defined in Table \ref{tab:selection_critera}. Finally, the remaining studies were read while considering both the inclusion and quality criteria.

\begin{table}[htp]
\caption{Inclusion and quality criteria for the selection process}
\begin{center}
\begin{adjustbox}{max width=\textwidth}
\begin{tabular}{|l|l|l|}\hline
Id & Criteria & Screening step\\\hline
IC 1 & \pbox{10cm}{The study's main concern is curriculum learning, dealing with noisy labels or road extraction systems}  & 1\\\hline
IC 2 & The study is presenting empirical results & 1\\\hline
IC 3 & The study preferably involve machine learning & 1\\\hline
QC 1 & The research has a clear aim & 2\\\hline
QC 2 & Is there an adequate description of related works? & 2\\\hline
QC 3 & How rigorously has the method or technique been tested? & 2\\\hline
QC 4 & Is there a future work section?
 & 2\\\hline
\end{tabular}
\end{adjustbox}
\end{center}
\label{tab:selection_critera}
\end{table}

\subsection{Other resources}
By conducting a \ac{SLR}, a number of relevant studies were identified. Additional literature was discovered by finding works citing the \ac{SLR} papers on Google Scholar, and reading survey papers on road extraction systems \citep{Mena_GIS_state_of_the_art} \citep{Trinder_towards_automation} and label noise \citep{Frenay_label_noise_survey}. The surveys provided a comprehensive overview of these topics and enabled further identification of relevant literature.