\section{Background discussion}
\label{sec:backgroundDiscussion}
This chapter has given a brief introduction to several topics. This include road extraction systems based on machine learning, convolutional neural networks, curriculum learning and label noise. In this section these topics will be summarised. \\

Manual labelling of aerial imagery for map purposes is a laborious task. Furthermore, the surface is always changing. These changes should preferable be reflected in the map data in a timely manner. Automatic object extraction has therefore become an compelling area of research.\\

The increasing spatial detail of aerial imagery is echoed in the choice of learning algorithm. The latest iteration consists of deep neural network, which are capable of learning complex decision boundaries, and extract suitable image features by learning. This is necessary, since the decreasing \ac{GSD}, have \todo{has} lead to more features being distinguishable from aerial imagery and increased the complexity of the data. The use of learning algorithms with a high model capacity, has also resulted in the use of larger datasets. Another trend is the increasing context window. For instance \cite{Mnih_roads_high_res_aerial_images} used a pixel neighbourhood of $64 \times 64$, to distinguish road from non-road pixels in a $16 \times 16$   center area. There are also many examples of systems that have combined various input features to increase accuracy.\\

Another compelling method for increasing performance of a road extraction system is the use of \ac{CRF}, or a post-processing neural network. In structured output, such as semantic segmentation,  smoothness between neighbouring predictions is an important consideration when performing a label assignment. For road extraction, these methods can reduce prediction artefacts such as disconnected roads, and can therefore result in a more even segmentation.\todo{To similar to future work?}\\

Training deep neural networks to extract roads from imagery require\todo{s?} large datasets. The labels in these datasets are often automatically generated from existing map data. Unfortunately, these labels are often affected by label noise, which can affect the performance of supervised learning. The types of label noise in aerial imagery have been identified by \cite{Mnih_aerial_images_noisy} as registration and omission noise.  \\

The first research question involves reducing the effect of inconsistent labels when training a classifier. The bootstrapping approach presented by \cite{Reed_noisy_labels_bootstrapping} will be evaluated for road detection. Whereas the loss functions presented by \citep{Mnih_aerial_images_noisy} require a modelling the noise distribution,  bootstrapping utilizes a convex combination between the classifier's prediction and the label. Furthermore, the method was tested for several datasets, and achieved good results. To test this method for road extraction, the robustness of this method can be evaluated on aerial image datasets with increasing levels of artificial omission and registration noise. Similar experiments of increasing noise levels have been conducted by \citep{Sukhbaatar_noisy_network_learning} and \citep{Reed_noisy_labels_bootstrapping}.\\

The studies involving curriculum learning demonstrated that a curriculum strategy which gradually introduce "harder" examples while training, resulted in a higher generalization accuracy and faster convergence. However, the curriculum strategies which \cite{Bengio_curriculumlearning} used to estimate the difficulty of each example were domain specific. This was rectified by \ac{SPL} \citep{Kumar_self_paced_learning}, where curriculum learning is internalized in the classifier. The model simultaneously estimates the difficulty of the examples and the loss. This method was extended by \citep{Lu_self-paced_learning_diversity}, where the diversity of the examples is \todo{are, but most likely is} also considered. The curriculum learning approach is also compliant with how humans prefer to teach, as demonstrated by \cite{Khan_human_teach}.\\

The second research question therefore investigates what effect a curriculum strategy can have on the road extraction system's precision and recall. Aerial image datasets often contain label noise, which would be considered "harder" in the context of curriculum learning. The learner would most likely create unnecessarily complex decision boundaries to fit the inconsistent examples. Postponing the introduction of these examples, could be beneficial for a road extraction system. An interesting curriculum strategy, mentioned by \cite{Bengio_curriculumlearning}, is creating an easy-to-hard ordering of samples according to how noisy the examples are. A potentially valid curriculum strategy could be to estimate the noisiness of examples, by measuring disagreement between labels and predictions.\\