\section{Background Discussion}
\label{sec:backgroundDiscussion}
This chapter has given a brief introduction to several topics. This includes road extraction systems based on machine learning, convolutional neural networks, curriculum learning and label noise. In this section, these topics will be summarized. \\

Manual labeling of aerial imagery for map purposes is a laborious task. Furthermore, the surface is always changing. These changes should preferable be reflected in the map data in a timely manner. Automatic object extraction has therefore become a compelling area of research.\\

The increasing spatial detail of aerial imagery is reflected in the choice of learning algorithm. The latest iteration consists of deep neural networks, which are capable of learning complex decision boundaries, and extract suitable image features by learning. This is necessary, since the decreasing \ac{GSD} has lead to more features being distinguishable from aerial imagery and increased the complexity of the data. The use of learning algorithms with high model capacity have also resulted in the usage of larger datasets. Another trend is the increasing context window. For instance, \cite{Mnih_roads_high_res_aerial_images} used a pixel neighborhood of $64 \times 64$ to distinguish road from non-road pixels in a $16 \times 16$ center area. There are also many examples of systems that have combined various input features to increase accuracy.\\

Furthermore, \ac{CRF} and post-processing neural networks have been used to increase the performance of various systems. In structured output learning, such as semantic segmentation, smoothness between neighboring predictions is an important consideration when performing a label assignment. For road extraction, these methods can reduce prediction artifacts such as disconnected roads, and can result in a more even segmentation.\\

Training deep neural networks to extract roads from images require large datasets. The labels in these datasets are often automatically generated from existing map data. Unfortunately, these labels are often affected by label noise, which can affect the performance of supervised learning. The types of label noise in aerial imagery have been identified by \cite{Mnih_aerial_images_noisy} as registration and omission noise.  \\

The first research question involves reducing the effect of inconsistent labels when training a classifier. The bootstrapping approach presented by \cite{Reed_noisy_labels_bootstrapping} was, therefore, evaluated for road detection in Chapter \ref{cha:ResearchAndResults}. Whereas the loss functions presented by \cite{Mnih_aerial_images_noisy} model the noise distribution,  bootstrapping utilizes a convex combination of the classifier's prediction and the label. Furthermore, the method was tested on several datasets, and achieved good results. To test this method for road extraction, the robustness of this method can be evaluated on aerial image datasets with increasing levels of artificial omission and registration noise. Similar experiments of increasing noise levels have been conducted in \citep{Sukhbaatar_noisy_network_learning} and \citep{Reed_noisy_labels_bootstrapping}.\\

The studies involving curriculum learning demonstrated that a curriculum strategy which gradually introduce ``harder" examples while training, resulted in an improved generalization accuracy and faster convergence. However, the curriculum strategies that \cite{Bengio_curriculumlearning} used to estimate the difficulty of each example were domain specific. This was rectified by \ac{SPL} \citep{Kumar_self_paced_learning}, where curriculum learning is internalized in the classifier. The model simultaneously estimates the difficulty of the examples and the loss. \cite{Lu_self-paced_learning_diversity} extended this method by also considering the diversity of the examples. The curriculum learning approach is also compliant with how humans prefer to teach, as demonstrated by \cite{Khan_human_teach}.\\

The second research question, therefore, investigates what effect a curriculum strategy can have on the road extraction system's precision and recall. Aerial image datasets often contain label noise, which could be considered harder in the context of curriculum learning. The learner would most likely create unnecessarily complex decision boundaries to fit the inconsistent examples. Postponing the introduction of these examples could be beneficial for a road extraction system. An interesting curriculum strategy, mentioned by \cite{Bengio_curriculumlearning}, is creating an easy-to-hard ordering of samples according to how noisy the examples are. A potentially valid curriculum strategy is to estimate the degree of noise in examples, by measuring disagreement between labels and predictions.\\