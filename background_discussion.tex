\section{Background discussion}
\label{sec:backgroundDiscussion}
This chapter has given a brief introduction to several topics. This include road extraction systems based on machine learning, convolutional neural networks, curriculum learning and label noise. In this section these topics will be summarised. \\

Manual labelling of aerial imagery for map purposes is a laborious task, and the surface landscape is always changing. These changes should preferable be reflected in the map data in a timely manner. Automatic object extraction has therefore become an compelling area of research.\\

Much more details. More features distinuishable from aerial imagery. Trend of powerful classifiers. Larger datasets. Increasing window size. CRF for smoother segmentation. Latest iteration deep neural networks for road extraction.

The large datasets -> automatically generated from existing map data. --> Result in datasets with certain types of label noise. Which can impact supervised learning. Registration and omission noise. Noise modelling, bootstrapping.

Curriculum learning. Compelling in the context of label noise. Inconsistent labelling is harder for a classifier to learn, considered hard. 
 The first research question involves reducing the effect of inconsistent labels when training a classifier. The bootstrapping approach presented by \cite{Reed_noisy_labels_bootstrapping} will be evaluated for road detection. Whereas the loss functions presented by \citep{Mnih_aerial_images_noisy} require a noise distribution model,  bootstrapping utilize a  convex combination between the classifier's prediction and the label. Furthermore, the method was tested for several datasets.\\

 An interesting improvement to this approach is to set a time-dependent parameter for how much the classifier's own prediction should be trusted. The parameter should, at the start of training, weight the convex combination more in favor of the label. While in later stages of training, the parameter can favor the classifier's predictions more, because they have become more accurate.\todo{Check if this was done already}\\

It is important to demonstrate that the approach offers robustness to label noise. This can be done by artificially introducing label noise in labels of road detection datasets. Omission noise can be simulated by removing a fraction of the ground truth in the label images, while registration noise can be introduced by shifting subsets of label images by a certain amount of pixels. Classifiers with the bootstrapping extension can be trained on these modified datasets, and compared to the performance achieved from baseline classifiers. The performances can then be compared at different noise levels, and reveal whether or not the technique offers any robustness towards noisy labels. Similar experiments have been conducted by \citep{Sukhbaatar_noisy_network_learning} and \citep{Reed_noisy_labels_bootstrapping} for a variety of datasets. \\

The second research question will investigate what effect a curriculum strategy can have on the road detection system's precision and recall. An interesting curriculum strategy, mentioned by \cite{Bengio_curriculumlearning}, is creating an easy-to-hard ordering of samples according to how noisy the examples are.

To evaluate curriculum learning, the road detection system could be trained, using both an unaltered training scheme and with curriculum learning. The results can then be compared, in order to determine whether curriculum learning can achieve better generalization for road detection.\\

\todo{CRF methods improves results quite a lot , but not a focus of this thesis. Fyll ut}
\todo{Aerial dataset, good dataset to test bootstrapping. Omission and registration errors, since dataset has not been hand-labelled for machine learning purposes. A real dataset, with an application area. }

In summary, the loss function of the road detection system will be modified, and an alternative training regime will be used for training. To measure the effect of these modifications, the system will be compared to a baseline system, and previous results achieved by similar systems.