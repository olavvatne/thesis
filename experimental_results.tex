\section{Experimental Results}
\label{sec:experimentalResults}

\subsection{Curriculum learning with aerial imagery}
\label{sec:results_curriculum_learning_aerial_imagery}
\begin{figure}[!ht]
\begin{subfigure}{0.48\textwidth}
\includegraphics[width=\linewidth]{figs/curr100/validation_loss_curve.png}
\caption{Comparison of validation loss.} \label{fig:curr100_loss}
\end{subfigure}
\hspace*{\fill} % separation between the subfigures
\begin{subfigure}{0.48\textwidth}
\includegraphics[width=\linewidth]{figs/curr100/validation_precision_recall.png}
\caption{Precision and recall comparison.} \label{fig:curr100_pr}
\end{subfigure}
\hspace*{\fill} % separation between the subfigures
\caption{E7 - Curriculum learning, with switch after 50 epochs. Models trained with 221600 examples. 10 runs, averaged} \label{fig:curr100}
\end{figure}


\begin{figure}[!ht]
\begin{subfigure}{0.48\textwidth}
\includegraphics[width=\linewidth]{figs/curr100/curriculum_loss_curves.png}
\caption{Curriculum loss.} \label{fig:curr100_loss2}
\end{subfigure}
\hspace*{\fill} % separation between the subfigures
\begin{subfigure}{0.48\textwidth}
\includegraphics[width=\linewidth]{figs/curr100/baseline_loss_curves.png}
\caption{Baseline loss.} \label{fig:curr100_epochs_baseline2}
\end{subfigure}
\hspace*{\fill} % separation between the subfigures
\caption{Loss for training, validation and test dataset. 221600 examples.} \label{fig:curr100_loss_epochs}
\end{figure}

\begin{figure}[!ht]
\begin{subfigure}{0.48\textwidth}
\includegraphics[width=\linewidth]{figs/E1/E1-lc-test.png}
\caption{Comparison of test loss} \label{fig:E1_curr_norway_loss}
\end{subfigure}
\hspace*{\fill} % separation between the subfigures
\begin{subfigure}{0.48\textwidth}
\includegraphics[width=\linewidth]{figs/E1/E1-pr-test.png}
\caption{Precision and recall comparisons.} \label{fig:E1_curr_norway_pr}
\end{subfigure}
\hspace*{\fill} % separation between the subfigures
\caption{E1 - Performance of curriculum learning at different thresholds, $D_{0}$ for Norwegian Roads Dataset Vbase} \label{fig:E1_curriculum_norway}
\end{figure}

\begin{figure}[!ht]
\begin{subfigure}{0.48\textwidth}
\includegraphics[width=\linewidth]{figs/E2/anticurr100_lc_comparison.png}
\caption{Comparison of test loss} \label{fig:E2_curr_mass_loss}
\end{subfigure}
\hspace*{\fill} % separation between the subfigures
\begin{subfigure}{0.48\textwidth}
\includegraphics[width=\linewidth]{figs/E2/anticurr100_pr_comparison.png}
\caption{Precision and recall comparisons.} \label{fig:E2_curr_mass_pr}
\end{subfigure}
\hspace*{\fill} % separation between the subfigures
\caption{E2 - Performance of curriculum learning and anti-curriculum learning for Massachusetts Roads Dataset} \label{fig:E2_curriculum_mass}
\end{figure}

\subsection{Bootstrapping for imagery with noisy labels}
\label{sec:results_bootstrapping}
\begin{figure}[!ht]
\begin{subfigure}{0.48\textwidth}
\includegraphics[width=\linewidth]{figs/E3/E3_lc_noise.png}
\caption{MSE loss} \label{fig:E3_boot_mass_loss}
\end{subfigure}
\hspace*{\fill} % separation between the subfigures
\begin{subfigure}{0.48\textwidth}
\includegraphics[width=\linewidth]{figs/E3/E3_pr_noise.png}
\caption{Precision and recall breakeven} \label{fig:E3_boot_mass_pr}
\end{subfigure}
\hspace*{\fill} % separation between the subfigures
\caption{E4 - Robustness of bootstrapping for increasing amount of label noise. Massachusetts Roads Dataset} \label{fig:E3_boot_mass}
\end{figure}

\begin{figure}[!ht]
\begin{subfigure}{0.48\textwidth}
\includegraphics[width=\linewidth]{figs/E5/E5_lc_noise.png}
\caption{MSE loss} \label{fig:E3_boot_norway_loss}
\end{subfigure}
\hspace*{\fill} % separation between the subfigures
\begin{subfigure}{0.48\textwidth}
\includegraphics[width=\linewidth]{figs/E5/E5_pr_noise.png}
\caption{Precision and recall breakeven} \label{fig:E3_boot_norway_pr}
\end{subfigure}
\hspace*{\fill} % separation between the subfigures
\caption{E4 - Robustness of bootstrapping for increasing amount of label noise. Norway Roads Dataset} \label{fig:E3_boot_norway}
\end{figure}

\subsection{Road detection system}
\label{sec:results_road_detection_system}
Nonetheless, the center image in Figure \ref{fig:result} illustrates the current performance of the system. In this particular test image, the model is able to identify the majority of the roads present, except for a small gravel road on the right side of the image. There are also a lot of prediction errors, such as roads being disconnected, and prediction artefacts in the forest areas. An interesting observation is that the model also correctly predicts small private roads leading up to houses present in the image. Furthermore, the model detects construction roads in the upper left corner. Since these roads are not present in the label image, the model is penalized for these predictions by the cross-entropy loss function.\\

\begin{figure}
\begin{subfigure}{0.48\textwidth}
\includegraphics[width=\textwidth]{figs/E6/E6-label.jpg}
\caption{Label image.} \label{fig:E6_label_iamge}
\vspace{0.5cm} % separation vertically between the subfigures
\end{subfigure}
\hspace*{\fill} % separation between the subfigures
\begin{subfigure}{0.48\textwidth}
\includegraphics[width=\textwidth]{figs/E6/E6-image.jpg}
\caption{Aerial image.} \label{fig:E6_aerial_image}
\vspace{0.5cm} % separation vertically between the subfigures
\end{subfigure}

\begin{subfigure}{0.48\textwidth}
\includegraphics[width=\textwidth]{figs/E6/E6-pred.jpg}
\caption{Model predictions.} \label{fig:E6_model_predictions}
\end{subfigure}
\hspace*{\fill} % separation between the subfigures
\begin{subfigure}{0.48\textwidth}
\includegraphics[width=\textwidth]{figs/E6/E6-hit.jpg}
\caption{Prediction hit and miss image.} \label{fig:E6_hit_iamge}
\end{subfigure}
\caption{E6 - Example of model's road detection performance. The aerial image is part of the test set in Massachusetts Roads Dataset} \label{fig:E6_performance}
\end{figure}


\todo[inline]{Display results in suitable representation.}
\todo[inline]{Choose what to present}
\todo[inline]{Present results for each experiment}
\todo[inline]{Present performance of system }
\todo[inline]{Or put this in it's own chapter}
\todo[inline]{Avoid drawing grand conclusions. Only what your data can support}
\todo[inline]{Study tables graphs for unusual things that might raise questions with the reader}