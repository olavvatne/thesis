% (C) Anders Kofod-Petersen
\documentclass[a4paper]{book}
\usepackage{listings}
\usepackage{todonotes}
\presetkeys{todonotes}{color=blue!20, bordercolor=white}{}
\usepackage[printonlyused]{acronym}
\usepackage[english]{babel}	
					% Correct English hyphenation
\usepackage{subcaption}
\usepackage[utf8]{inputenc}						% Allow for non-English letters
\usepackage{graphicx}							% To include graphics
\usepackage{placeins}
%Float barriers
\usepackage{natbib}								% Correct citations
\usepackage{pbox}								% Adjust new line for cells
\usepackage{adjustbox}
%\usepackage{fancyheadings}						% Nice header and footer
\usepackage[linktocpage,colorlinks]{hyperref}			% PDF hyperlink
\usepackage{geometry} 							% Better geometry
%\usepackage[center]					% For cropping documents

% B5 (uncomment to convert to B5 format)
 \geometry{b5paper}

% Author
% Fill in here, and use commands in the text. 
\newcommand{\thesisAuthor}{Olav Kåre Vatne}
\newcommand{\thesisTitle}{Road detection in aerial images}
\newcommand{\thesisType}{master project}
\newcommand{\thesisDate}{spring 2016}

% PDF info
\hypersetup{pdfauthor={\thesisAuthor}}
\hypersetup{pdftitle={\thesisTitle}}
\hypersetup{pdfsubject={\thesisType}}
\hypersetup{linkcolor=black}
\hypersetup{citecolor=black}
\hypersetup{urlcolor=black}

%Fancy headings
%\pagestyle{fancy}
%\pagestyle{fancyplain}
%\renewcommand{\chaptermark}[1]{\markboth{#1}{}}
%\renewcommand{\sectionmark}[1]{\markright{#1}{}}
%\lhead[\fancyplain{}{\thepage}]{\fancyplain{}{\let\uppercase\relax\leftmark}}
%\rhead[\fancyplain{}{\let\uppercase\relax\rightmark}]{\fancyplain{}{\thepage}}
%\chead[\fancyplain{}{}]{\fancyplain{}{}}
%\lfoot[\fancyplain{}{}]{\fancyplain{}{}}
%\cfoot[\fancyplain{}{}]{\fancyplain{}{}}
%\rfoot[\fancyplain{}{}]{\fancyplain{}{}}

% Citation format
\bibliographystyle{apalike}
\bibpunct{[}{]}{;}{a}{,}{,}

\begin{document}

%Title page (This is generate automatically from the commands above)
\begin{titlepage} 
\noindent {\large \textbf{\thesisAuthor}}
\vspace{2cm}

\noindent {\Huge \thesisTitle}
\vspace{2cm}

\noindent \thesisType, \thesisDate 
\vspace{2cm}

\noindent Artificial Intelligence Group\\ Department of Computer and Information Science\\ Faculty of Information Technology, Mathematics and Electrical Engineering\\

\vfill
\begin{center}
\includegraphics[width=3cm]{figs/NTNUlogo.pdf}
\end{center}
\end{titlepage}

\thispagestyle{empty}

\cleardoublepage

\frontmatter

\clearpage

\section*{Abstract}
\todo[inline]{Abstract- Sales pitch}
\todo[inline]{field of research, brief motivation for the work, what the research topic is, the research approach(es) applied. contributions}
\todo[inline]{Half a page of text - no lists tables or figures}

\clearpage

\section*{Preface}
\vspace{1cm}
\todo[inline]{Preface -reword}
This project is the authors master thesis at the Department of Computer and Information
Science, Norwegian University of Science and Technology. \\

Special thanks to supervisor Professor Keith Downing, at the Department of Computer
and Information Science, Norwegian University of Science and Technology, for his
helpful advice and guidance throughout this project. \\

I would also like to thank Ingvild Olaussen for tirelessly correcting my grammar, and make me sound somewhat eloquent throughout this thesis.


\vfill

\hfill \thesisAuthor

\hfill Trondheim, \today

\clearpage

\section*{List of Abbreviations}
\vspace{1cm}
\begin{acronym}
\acro{CNN}{convolutional neural network}
\acro{CPU}{central processing unit}
\acro{CRF}{conditional random fields}
\acro{GIS}{geographic information system}
\acro{GPU}{graphics processing unit}
\acro{GSD}{ground sampling distance}
\acro{SGD}{stochastic gradient descent}
\acro{SLR}{structured literature review}
\acro{SPL}{self-paced learning}
\acro{SPLD}{self-paced learning with diversity}
\acro{SVM}{support vector machines}
\end{acronym}



\tableofcontents

\listoffigures

\listoftables


\mainmatter



\chapter{Introduction}
\label{cha:Introduction}
In this report, the state of the art for road extraction is explored. This particular area is a part of the field of Photogrammetry. Additionally, research into topics such as curriculum learning, and how to handle label noise have been conducted. This research will be useful in the development of a road detection system.\\

This chapter aims at giving an introduction to the report. Section \ref{sec:BackgroundAndMotivation} outlines the background and motivation, and in Section \ref{sec:Goals and Research Questions} goals and research questions are presented. Research methods are described in Section \ref{sec:researchMethod}, Section \ref{sec:IntroContributions} presents the thesis' contributions \todo{Weird way of putting it} and Section \ref{sec:thesisStructure} gives an overview over the thesis structure.

\section{Background and Motivation}\label{cit}
\label{sec:BackgroundAndMotivation}
Photogrammetry, or remote image sensing, is a field occupied with obtaining measurements about object or areas from overhead imagery, typically captured from an airplane or satellite. Common tasks in remote image sensing are land cover classification or road extraction. In land cover classification, each pixel of an aerial image is assigned a land cover label, such as grass, water, building or road. Road extraction is a binary land cover classification task where each pixel is categorized as being a road or a non-road pixel.\\


Aerial and satellite images contain a variety of different features. Identification of these features is often done by a human expert, which can be expensive, both in terms of cost and time. Additionally, there is an increasing availability of high-resolution overhead imagery, which makes a machine learning approach for automatic land cover classification compelling. \\

Feature extraction from aerial images is a non-trivial task because of the complexity presented by images. An array of pixel intensities might represent natural land covers such as terrain, vegetation, or artificial objects such as roads or buildings. Each type can look very different in terms of shape and texture. Additionally, aerial images are exposed to different variations of illumination such as objects casting shadows or changes in brightness. There is also the issue of occlusion. Roads can, for example, be partly occluded by cars and trees. The result is that extraction of information from aerial or satellite images can be challenging for an automatic extraction system. \\

A machine learning approach to land cover classification is typically formulated as a semantic segmentation task. Given an aerial image as seen in Figure \ref{fig:aerialimage}, an algorithm should segment the image into disjoint regions, such as water, road, building, grass or tree. Alternatively, the algorithm could do a binary classification of the image, where each pixel is either a member or non-member of a class.\\

\begin{figure}[t]
\begin{center}
\includegraphics[width=0.8\columnwidth]{figs/aerial_image.png}
\caption[Aerial image]{An aerial image captured above NTNU.}
\label{fig:aerialimage}
\end{center}
\end{figure}

A \ac{CNN} is a special variant of a neural network, where connectivity between units have been constrained and parameter sharing is employed. This will reduce the number of parameters in the model. \ac{CNN}s can, therefore, have many hidden layers, which enables the network to learn a hierarchical representation of the input data. By having a large dataset and conducting normal back-propagation, the network can learn to extract informative features from raw pixel values. \\

This might be well suited for a road extraction system, where there are an abundance of aerial images available covering large areas. Labels can easily be generated for these areas from digital maps stored in a \ac{GIS} database. However, a large issue with utilizing aerial images for training a machine learning algorithm is the presence of noise in the labels. For most purposes, maps can be created without pixel level accuracy and still retain their usefulness. The result is that datasets created from digital maps have some degree of label noise, which can have negative consequences for the accuracy achieved by a machine learning algorithm. \cite{Mnih_aerial_images_noisy} have identified two types of label noise present in aerial images: Omission and registration noise. The former occurs when an object in an aerial image is missing, and the latter happens when there is a misalignment between the object in the image and in the ground truth of the label.\\

One way of reducing the impact of noisy labels, is by modifying the loss function. Cross-entropy loss assumes that the labels are correct, which results in noisy labels incorrectly penalizing an accurate classifier. The bootstrapping method proposed by \citep{Reed_noisy_labels_bootstrapping}, utilizes the classifier's implicit knowledge about the task, through incorporating the classifier’s predictions in a convex combination with the labels. The quality of these modified targets improves as the classifier's accuracy increases, which is why the method is called bootstrapping.\\

Another compelling way of improving generalization accuracy of a machine learning algorithm, is the use of curriculum learning \citep{Bengio_curriculumlearning}. The method involves organizing the dataset according to a curriculum, where examples are sorted based on a criterion of ``easiness". The examples considered ``easy" are presented early on in the optimization process, whereas ``harder" examples are presented later on.


\section{Goals and Research Questions}
\label{sec:Goals and Research Questions}
In this section, the goal and research questions of this thesis are presented, as well as a brief motivation for each research question.

\begin{description}[ style=nextline, leftmargin=1.5em, rightmargin=1.5em]
\item[Goal statement:]{\it The goal of this thesis is to create a convolutional neural network that can extract roads from aerial images.}
\end{description}

The thesis will investigate how to further improve road extraction by considering the two research question defined below. A system consisting of a convolutional neural network will be created, and experiments will reveal how well the system performs.

\begin{description}[ style=nextline, leftmargin=1.5em, rightmargin=1.5em]
\item[Research question 1:]{\it Does the bootstrapping loss function give a significant improvement of precision and recall for datasets with noisy labels?}
\end{description}

Because of the costs involved in creating accurate datasets, the thesis will look at techniques to reduce the effect of inconsistent labelling. For datasets related to aerial images, it is common to find omission and registration error. Small and private roads are often unmarked on maps, and roads might be incorrectly placed on them.

\begin{description}[ style=nextline, leftmargin=1.5em, rightmargin=1.5em]
\item[Research question 2:]{\it How can curriculum learning improve results in deep learning, and does this improve precision and recall for aerial images?}
\end{description}

The thesis will also investigate the benefits of curriculum learning for a machine learning algorithm. By sorting examples from easy to hard, the learner can potentially be guided to a more advantageous area of parameter space and result in the learner finding a better local minima, as well as reducing the time of convergence. This can be beneficial for deep learning, which often involves optimization of a lot of parameters. The challenge is to find an ordering criteria that is applicable for aerial images. 


To address the research questions outlined in Section \ref{sec:Goals and Research Questions}, a \ac{CNN} has been developed. An aerial image dataset containing ground truth for roads have been acquired from \citep{MnihThesis}, and is publicly available under the name Massachusetts Roads Dataset. Additionally, a new dataset containing parts of the Norwegian road network has been created using aerial images and road center-line vectors provided by Kartverket.\\   

These datasets will be used for training and testing the performance of the system, as well as evaluating the research questions. Furthermore, by utilizing a publicly available dataset, the system presented in the thesis can be compared to the performance of other similar systems \citep{Mnih_aerial_images_noisy}\citep{MnihThesis}\citep{saito_building_and_roads}.  

\section{Contributions}
\label{sec:IntroContributions}
The thesis' main contribution to the field of machine learning  is the examination of different approaches that can reduce the impact of inconsistent labelling, and experiments conducted on semantic road labelling task which has a high rate of naturally occurring inconsistent labelling. Experiments demonstrated that curriculum learning improved the generalization accuracy of a deep neural network, trained with real-world datasets. Furthermore, the thesis shows that a curriculum teacher based on estimating inconsistency between a model prediction and a label can be effective for curriculum learning.

\section{Thesis Structure}
\label{sec:thesisStructure}
The thesis is divided into five chapters. In Chapter \ref{cha:TheoryAndBackground}, sections describing the background theory, the structured literature review, and related work can be found. Chapter \ref{cha:architectureAndModel} outlines the methods and implementation, as well as presenting details about the datasets. Experiments, results and discussion can be found in Chapter \ref{cha:ResearchAndResults}. The final chapter, Chapter \ref{cha:evaluationAndConclusion}, concludes the thesis by summarising the results, outlining future work and presenting contributions.



\chapter{Background Theory and Motivation}\label{T-B}
\label{cha:TheoryAndBackground}

\section{Background Theory}
\label{sec:background_theory}
\subsection{Convolutional Neural Networks}
A \ac{CNN} is a special kind of neural network, and it was one of the first deep learning models to perform well in commercial applications. A \ac{CNN} is loosely based on principles drawn from neuroscience. According to \cite{Bengio_deep_learning_book},  local connectivity and parameter sharing are properties characteristic for \ac{CNN}s. These properties and the architecture of \ac{CNN}s will be explored further below.

\subsubsection{Convolution}
In mathematics, convolution is a mathematical operation on two real-valued functions that express the amount of overlap of one function as it is shifted over another function. For machine learning applications, the data is usually discretized. The operation is therefore a discrete summation over the data, and is used to calculate the weighted sum between the activations and the connection weights in a \ac{CNN}. The discrete convolution operation without kernel-flipping:

$$ (x*w)(t) = \sum\limits_{a=-\infty}^\infty x[a]w[t+a]$$ 

For aerial images we extend the convolution operation to two dimensions, and limit the summation to a finite number of pixels. To convolve an image I, a two-dimensional kernel K containing the weights is shifted across the image:  

$$ (I*K)[i,j] = \sum\limits_{m}\sum\limits_{n} I[i+m, j+n]K[m,n]$$ 

This operation is  visualized in figure \ref{fig:convolution} where a $2 \times 2$ kernel of weights is convolved with a $3 \times 3$ matrix of input values, and produces $2 \times 3$ outputs. 

\begin{figure}[t]
\begin{center}
\includegraphics[width=0.8\columnwidth]{figs/convolution.png}
\caption[Convolution example]{Example of 2D convolution without kernel-flipping.}
\label{fig:convolution}
\end{center}
\end{figure}


\subsubsection{Local Connectivity}
In a traditional neural network, each layer is typically fully connected. Each unit has connections to every unit in the previous layer. In a \ac{CNN}, however, a unit interacts only with a small region of units in the previous layer. This region is often referred to as the unit's local receptive field. This kind of local connectivity can be very practical for high-dimensional data, such as images where meaningful features can be extracted using only a small area of the total image. \\

Local connectivity can be achieved by using a small kernel as seen in Figure \ref{fig:convolution}. Instead of each unit being connected to all inputs, the unit only depends on a $2 \times 2$ input region. \\

If there are $m$ inputs and $n$ units, a matrix multiplication for a fully-connected network would require $m\times n$ parameters, as well as having a runtime of $O(m\times n)$. By using a kernel we limit the number of connections each unit may have to k. This requires only $k\times n$ parameters and a runtime of $O(k\times n)$. For image applications, the kernel size can be relatively small and still achieve good results, which can give big improvements in efficiency.

\subsubsection{Parameter Sharing}
The number of model parameters are \todo{is?} further reduced by using parameter sharing. Each weight in the kernel is applied to every position of the input. In contrast, a neural network which is fully connected will have a separate weight for every connection. This can be redundant for high-dimensional data, where most of the features are localized. In images, for example, an important feature to extract are edges. A kernel with weights that are good at detecting edges at one location, will be equally good at detecting them in other locations. \\

The use of parameter sharing further reduces the storage requirement to $k$ parameters. Usually, one kernel per layer is not enough, so several kernels with tied weights convolve the input. The layer will then produce output activations for different features. The output of several kernels are \todo{is?} often referred to as feature maps.

\subsubsection{Pooling}
The pooling function is another operation typically associated with \ac{CNN}. A pooling function modifies the output of a layer in some way. It replaces a rectangular region of the output by a single value that has been determined by a summary operation. A common pooling function is the max pooling operation, which outputs the maximum within a rectangular neighbourhood. The reason for utilizing pooling is that it helps the representation become invariant to small translations in the input. For example, a network created to classify whether an image depicts a cat or not will benefit from pooling, since the location of the cat in the picture is irrelevant. For tasks where the location of a feature is important, such as semantic segmentation, applying pooling should be done with restraint. Additionally, pooling reduces the number of input parameters for the next layer.

\subsubsection{Layer Structure}
A typical convolutional layer in a network consists of three stages. First, convolution sums the weighted inputs for every unit in the layer. Second, an activation function is applied to the resulting values. The \ac{ReLU} is a popular choice, and outputs either 0 or the weighted sum, depending on which is biggest: $f(x) = max(0, x)$. Finally, the pooling function modifies the output of the layer. 

\subsubsection{Network architecture}
A \ac{CNN} usually consists of both convolutional layers and fully-connected layers. The input layer and the initial hidden layers are convolutional layers, with fully connected layers attached at the end. Figure \ref{fig:conv} shows a convolutional neural network configuration. The input layer and the two first hidden layers are convolutional layers. During training, the kernel weights for these layers are adjusted by back-propagation. Each feature map defines a set of kernel weights that are applied to all input pixels or activations. Usually, a \ac{CNN} will reduce the necessity of feature engineering because it learns what suitable features to extract from input data. In images, a \ac{CNN} is able to learn from raw pixel values without the use of feature extraction techniques found in computer vision.


\begin{figure}[t]
\begin{center}
\includegraphics[width=1\columnwidth]{figs/conv_diagram.png}
\caption[Convolutional neural network]{Convolutional neural network. }
\label{fig:conv}
\end{center}
\end{figure}
\label{sec:convolutional_networks_background}

\subsection{Label Noise}
\label{sec:background_label_noise}
There are several reasons for the presence of inconsistent labels in real-world datasets. For instance, the labeller was presented with insufficient information, or the dataset was automatically generated from a source with poor quality labels. Additionally, the samples could be ambiguous and therefore hard to label correctly by a human expert. Label noise, can in many cases, lead to negative consequences for a classifier. This can include reduced accuracy, increased model complexity, and more samples required for learning a target concept. Approaches for dealing with noisy labels can generally be divided into three groups: Data cleansing methods, noise-robust models, and noise-tolerant algorithms \citep{Frenay_label_noise_survey}. These three groups are presented below.  \\

%\subsubsection{Types of noisy labels}
%Based on the noise distribution, three types of label noise can be identified according a extensive survey on label noise by \citep{Frenay_label_noise_survey}. The different types can be viewed in Figure ??\todo{Figur}. X is a vector of features or the data. For all types the observed variable Y* is assumed to depend the true label Y.  The E is a binary variable indicating if an labelling error has occurred. 

%Noisy completely at random (NCAR) model.

%Noisy at random (NAR) model. 

%Noisy not at random (NNAR) model


\subsubsection{Data Cleansing Methods}

Data cleansing methods are filtering techniques applied to the training data in order to remove noisy samples before training. Noisy labels are first identified and then either relabelled or removed. An obstacle encountered by these methods is that harmful mislabelled samples can be difficult to distinguish from informative, but hard samples. Another problem is that filtering often relies on classifier predictions to automatically identify mislabelled samples. Such filtering techniques also run the risk of removing too many samples from the training set, which can also cause harm to the accuracy. Voting ensembles of several classifiers have been suggested to further improve classification filtering.\\ 

Another filtering technique is to simply remove the class label of samples deemed suspicious, and employ semi-supervised learning. This way, the distribution of samples are preserved while simultaneously reducing the consequences of inconsistent labels.\\


\subsubsection{Noise-robust Models}
Noise-robust models are algorithms that are naturally robust against label noise. Many algorithms have been shown to be less sensitive to label noise than others, especially to small amounts of label noise. This approach requires no noise modelling nor cleansing of the training set beforehand, because the algorithm is assumed to offer some robustness to mislabelled samples.\\

 Algorithms that utilize regularization techniques to avoid overfitting, can be considered more robust to label noise. This can include convolutional networks that utilize regularization schemes such as dropout or weight decay. For ensemble methods, bagging often gives better results than boosting when faced with noisy labels \citep{Dietterich_boosting_bagging}. The boosting algorithm AdaBoost, for example, combines many weak classifiers by iteratively re-weighting the training set to target samples the previous classifier had trouble predicting. Because mislabelled samples can be harder to predict, AdaBoost tends to put larger emphasis on mislabelled samples in later stages of learning, which can lead to increased sensitivity to label noise. In bagging methods, however, different subsets of the training data are used to create a diverse set of classifiers that are employed in a voting scheme. In this case, mislabelled samples can impact the performance positively, due to the increased variability in the classifiers.   


\subsubsection{Noise-tolerant Algorithms}
In noise-tolerant approaches, existing algorithms are modified to be more robust towards label noise. This is often done by explicitly modelling a noise model during training. This way, a classifier learns to classify samples according to their true uncorrupted label, instead of the observed noisy label. Typically, the noise distribution and the model parameters are estimated simultaneously when training the classifier. \\

Techniques that incorporate label noise tolerance, such as particle competition, noise model estimation, bootstrapping, and co-training, will be further discussed in Section \ref{sec:related_works}.

\subsection{Curriculum Learning}
Curriculum learning is inspired by how humans learn, and that learning typically is highly organized. For instance, by the use of a curriculum in educational institutions. Easier concepts tend to be introduced first. In terms of machine learning, this means presenting the classifier with easier samples first while training. To do so, a curriculum strategy has to be defined, which sorts the training set from easy to hard. Samples that are not near the decision boundary could be considered easy, for instance. Utilizing curriculum learning might lead to a faster convergence time, and the algorithm reaching a better local minima. Different works show that curriculum learning can achieve better generalization for many tasks \citep{Bengio_curriculumlearning} \citep{Kumar_self_paced_learning} \citep{Lu_self-paced_learning_diversity}.\\

A challenge for curriculum learning is defining a sorting measure that enables a curriculum strategy of gradually introducing harder training samples to the learner. This issue, and works related to curriculum learning, is further explored in Section \ref{sec:related_works}.\\

\subsection{Road Extraction by Machine Learning}
Road extraction is a part of the field of Photogrammetry. The field involves technology for map production and measurements of objects in images. As digital acquisition systems for capturing aerial images have become commonplace, the availability of high resolution aerial images have increased. Coupled with the increasing need for detailed spatial information in \ac{GIS} databases and production of digital maps, a lot of approaches for automatic object extraction from aerial imagery has been suggested. A reliable and accurate detection system could be beneficial in terms of increased levels of details, while reducing the cost associated with map production.\\

There are three distinctive approaches for extracting objects from aerial imagery. Manual, semi-automatic and automatic methods. The semi-automatic approach integrates computer vision techniques and machine learning into the workflow of skilled human labellers. For instance, Google's Ground Truth project employs skilled operators which utilize advanced software tools to improve the accuracy of Google's map products \citep{Ground_truth}.\\

To do automatic road extraction, supervised learning is often employed. This requires a training dataset of aerial image examples and labels. The labels for road extraction are usually binary images that shows the ground truth of roads. Creating the label images by manually labelling aerial imagery would be prohibitively expensive, which is why ground truth labels are often generated from existing map data. Figure \ref{fig:background_dataset_example} shows an example of aerial image and label typically found in a road extraction dataset.\\

\begin{figure}
\begin{subfigure}{0.40\textwidth}
\includegraphics[width=\linewidth]{figs/background_theory_example_data.png}
\caption{Aerial image} \label{fig:background_dataset_example_data}
\end{subfigure}
\hspace*{\fill} % separation between the subfigures
\begin{subfigure}{0.40\textwidth}
\includegraphics[width=\linewidth]{figs/background_theory_example_label.png}
\caption{Label image} \label{fig:background_dataset_example_label}
\end{subfigure}
\caption[Example from the Norwegian Roads Dataset]{Image and label example from the training set of the Norwegian Roads Dataset.} \label{fig:background_dataset_example}
\end{figure}

From these large training set images, smaller training set patches are extracted. The supervised learning algorithm is given a patch dataset $d$ containing $N$ training examples in the form $d=\{(s_1, m_1),...,(s_N, m_N)\}$, where $s_i$ is an aerial image patch and $m_i$ is the corresponding ground truth label. Examples of aerial image patches and labels can be found in Figure \ref{fig:examples_background}. The learning algorithm's task is to learn a suitable mapping from the input space of aerial imagery to the output space of road ground truth. For neural networks, this mapping is typically learned by minimizing the cross-entropy loss by gradient descent optimization.\\

\begin{figure}
\begin{center}
\includegraphics[width=1\columnwidth]{figs/examples.png}
\caption[Patch dataset examples]{Aerial image and label examples from a patch dataset. The patches originate from the Norwegian Roads Dataset. Each label indicates the road ground truth from the center location of the aerial patch.}
\label{fig:examples_background}
\end{center}
\end{figure}

 The resulting classifier has hopefully extracted some useful patterns from the training data, which enables it to generalize to the task of road extraction. This is verified by computing the \ac{MSE} on a test set, containing examples not seen during training. The machine learning approach for road extraction, should therefore be able to train algorithms from data which can predict the ground truth reasonable well for new unseen aerial image patches. \\

\subsection{Evaluation Metrics}
A common way to evaluate road extraction systems is by the quality measures,  correctness and completeness \citep{Wiedemann_road_evaluation}. These are closely related to precision and recall. Precision measures the fraction of true roads that are correctly detected, while recall is the fraction of predicted roads that are true roads. Because the label maps are not perfectly aligned with the images, it is also common to use a relaxed measure of precision and recall. This is accomplished by treating predicted road pixels within $p$ pixels of the the true road pixel as being correctly detected, and true roads within $p$ of a predicted road pixel as properly recalled. The slack parameter $p$ is often set to 3 pixels \citep{Mnih_roads_high_res_aerial_images}.



\section{Structured Literature Review}
\label{sec:slr}
The purpose of conducting a \ac{SLR} is to get an overview of the field of remote image sensing, as well as the research related to curriculum learning and dealing with noisy labels. The \ac{SLR} method has been chosen to investigate these topics, and provides a formal way of identifying the information available.

\subsection{Identification of Research}
This section outlines the strategy that was utilized to search for primary studies. By utilizing a search strategy, literature relevant to the defined research questions can be identified and collected. Search terms have been defined, as well as literature resources.

\subsubsection{Literature Resources}
\label{sec:literature_resources}
The following resources was searched in order to identify and collect relevant material:
\begin{itemize}
	\item ACM digital library
	\item IEEExplore
	\item ScienceDirect
	\item CiteSeer
	\item Springer Link
	\item ECCV (conference)
	\item NIPS (conference)
\end{itemize}

\subsubsection{Key Terms and Groupings}
Table \ref{tab:search_terms} describes the different word groups employed by the search strategy. Each group specifies a number of terms that are either synonyms or related to each other. Additional terms have been appended, as a result of a search strategy validation.


\begin{table}[htp]

\caption[The terms and groups]{The terms and groups.}
\begin{center}
\begin{adjustbox}{max width=\textwidth}
\begin{tabular}{+c | ^p{0.12\textwidth} ^p{0.14\textwidth} ^p{0.15\textwidth} ^p{0.19\textwidth} ^p{0.15\textwidth} ^p{0.12\textwidth} }\hline
\rowstyle{\bfseries}
 		& Group 1 & Group 2 & Group 3 & Group 4 & Group 5 & Group 6\\\hline
\textbf{T1} 	& Aerial images & Curriculum learning & Noisy labels & Neural network & Segmentation & Roads\\
\textbf{T2}	& Satellite images & Guided learning & Missing \mbox{labels} & Convolutional neural network & Classification & \\
\textbf{T3} 	& Remote sensing & Example ordering & Semi-supervised & Machine \mbox{learning} & & \\
\textbf{T4} 	& Images & & Noisy data (appended) & Deep \mbox{neural} networks (appended) & & \\\hline
\end{tabular}
\end{adjustbox}
\end{center}
\label{tab:search_terms}
\end{table}

\subsubsection{Search Strategy}
Based on Table \ref{tab:search_terms} several search expressions were devised. Each of these expressions were run for each of the resources listed in the literature resources section. 

\begin{itemize}
	\item (aerial images OR satellite images) AND (segmentation OR classification)
	\item (remote sensing OR aerial images) AND (noisy labels OR missing labels OR semi-supervised OR noisy data)
	\item (neural network OR machine learning OR convolutional neural networks) AND (aerial images OR satellite images)
	\item (curriculum learning OR guided learning) AND (machine learning OR neural networks OR deep neural networks OR  convolutional neural network) 
	\item (segmentation OR classification) AND (roads)
	\item (Noisy labels OR missing labels OR semi-supervised OR noisy data) AND (machine learning OR neural network OR convolutional neural network)
	\item (aerial) AND (noisy labels OR missing labels OR semi-supervised OR noisy data) AND roads
\end{itemize}

\subsection{Selection Process}
The search expressions resulted in a large number of hits for each resource. The top 15 results for each expression were stored, and a  selection process was created to reduce the over 400 studies to a more manageable number. At first, the title of each study was evaluated. If the title seemed unrelated to the research goal or the research questions found in Section \ref{sec:Goals and Research Questions}
the study was removed. Then the title and the abstract were evaluated using the inclusion criteria defined in Table \ref{tab:selection_critera}. Finally, the remaining studies were read while considering both the inclusion and quality criteria.

\begin{table}[htp]
\caption[Inclusion and quality criteria for the selection process]{Inclusion and quality criteria for the selection process.}
\begin{center}
\begin{adjustbox}{max width=\textwidth}
\begin{tabular}{+l ^p{8cm} ^r}\hline
\rowstyle{\bfseries}
Id & Criteria & Screening step\\\hline

IC 1 & The study's main concern is curriculum learning, dealing with noisy labels or road extraction systems  & 1\\
IC 2 & The study is presenting empirical results & 1\\
IC 3 & The study preferably involve machine learning & 1\\
QC 1 & The research has a clear aim & 2\\
QC 2 & Is there an adequate description of related works? & 2\\
QC 3 & How rigorously has the method or technique been tested? & 2\\
QC 4 & Is there a future work section?
 & 2\\\hline
\end{tabular}
\end{adjustbox}
\end{center}
\label{tab:selection_critera}
\end{table}

\subsection{Other Resources}
By conducting a \ac{SLR}, a number of relevant studies were identified. Additional literature was discovered by finding works citing the \ac{SLR} papers on Google Scholar, and reading survey papers on road extraction systems \citep{Mena_GIS_state_of_the_art} \citep{Trinder_towards_automation} and label noise \citep{Frenay_label_noise_survey}. The surveys provided a comprehensive overview of these topics and enabled further identification of relevant literature.

\section{Related work}
\label{sec:related_works}
\subsection{Road Extraction Systems}
There is a large amount of literature regarding proposed methods for automatic road extraction systems. This includes segmentation, edge detection, knowledge based methods, fuzzy classification methods, and region growing methods. For a thorough review of different road extraction systems, see \citep{Trinder_towards_automation}
\citep{Mena_GIS_state_of_the_art}. The main aim of this section is to present approaches based on machine learning.

\subsubsection{Types of Aerial Imagery}
Aerial and satellite imagery are captured using a whole range of sensors. A lot of approaches found in remote sensing are developed for certain types of sensor data. Below is a list of different types of aerial imagery:
\begin{itemize}
 \item Monochromatic (single channel or greyscale) images 
 \item Infrared band
 \item Color images (Red, green and blue channel)
 \item Hyper-spectral images
 \item Synthetic aperture radar images (SAR)
 \item Laser images (LIDAR)
 \end{itemize}

The focus of this review is approaches for color images, which is the most common type of aerial imagery.

\subsubsection{Machine Learning}
In the previous two decades, the availability of high resolution images covering large areas have increased. These images can have a resolution of around 1 square meter per pixel or higher. At this resolution, finer details such as cars, buildings and trees can be distinguished. Having a higher resolution for images also result in much more variability in terms of shape, texture and illumination. Machine learning algorithms that can learn highly non-linear decision boundaries have therefore become more common for aerial imagery applications. To successfully discriminate between object classes, more spatial context have been used to create a richer feature representation, as well as more data being used for training. Additionally, structured prediction methods such as \ac{CRF}, have become popular for smoothing in semantic segmentation applications.


\subsubsection{Classifiers}
High-resolution aerial imagery has created a need for more sophisticated classifiers. A classifier must encode knowledge of shape and context in order to discriminate between similar objects. In a road extraction system for high resolution imagery, the classifier should, for example, be able to distinguish between roads and grey rooftops. The classifier is therefore required to learn highly non-linear decision boundaries. This can include \ac{SVM}, ensemble methods and deep neural networks.\\

In \cite{Mayer_road_test} six road extraction approaches were compared on both aerial images and satellite images. The approaches were based on image processing techniques, especially line detection. Many of the approaches rely on characteristics specific to roads, such as identifying parallel lines in images. In addition, some approaches utilized fuzzy classification or unsupervised clustering. Both scenes from urban and rural areas were used for the comparison. The authors defined a minimum of 0.6 and 0.75 in precision and recall, in order for a raod extraction system to be of any practical use. In summary, most of the approaches performed well for images with limited complexity, such as rural areas. None of the methods performed above the defined threshold for images containing suburban or urban scenes. The low performance for urban areas reinforces the need for classifiers that can learn complex decision boundaries.\\

A hybrid approach for road extraction using \ac{SVM} and image processing techniques was proposed by \cite{Song_road_extraction_svm}. First, a \ac{SVM} classifies images into a road and a non-road set. Second, the images in the road set are segmented into homogeneous areas by utilizing the region growing technique. \\

The \ac{SVM} did not perform sufficiently well for areas that appear similar to roads, especially for urban areas, with structures such as parking areas and roof tops. Therefore, they extract shape descriptions from the segmented regions, and exploit road characteristics to remove areas not corresponding to roads. This is done by a threshold operation on shape descriptions such as smoothness and density.\\

 The approach was tested experimentally on IKONOS satellite images. The approach performed well. However, road segments obscured by shadows or overhanging trees were a problem, as well as narrow roads and intersections.\\

Ensemble methods have also been applied to tasks involving aerial imagery. In \citep{Kluckner_semantic_height}, randomized forest was used for land cover classification on high-resolution imagery. Training a randomized forest involves training several binary decision trees on subsets of the training data. The result is an ensemble of weak classifiers that together can provide robust and accurate predictions. \cite{Dollar_supervised_edge} proposed a supervised edge detection method, which was tested for road detection. This method trains a boosted tree classifier, which is similar to a decision tree, except that boosted classifiers are used to split the data at each node in the tree.\\

\cite{Mnih_roads_high_res_aerial_images} proposed an automatic road extraction approach for large real-world datasets. The system consists of a neural network with millions of weights that is trained on a large dataset of aerial images. A \ac{GPU} was utilized to train the network.  \\

They formulated detection of road pixels from aerial images as a patch-based semantic segmentation problem. 
The goal of the model is to predict whether or not pixels belong to a road class, given an image patch. This is achieved by having the neural network model the distribution:

 $$p(N(M(i,j), w_m) \mid N(S(i,j), w_s),$$ 
 
\noindent where $S$ is an aerial image and $M$ is a corresponding road label image. $M(i,j) = 1$ if $S(i,j)$ is a road pixel and 0 otherwise.  $N(I(i,j), w)$  denotes a $w \times w$ large patch of pixels centered at location (i,j) of a large image $I$. Using smaller image patches instead of entire images for modelling the distribution, limits the image context which the model use to make predictions. This approach is less computational expensive, and by retaining a relatively large $w_s \times w_s$ can still provide enough image context to create a competent road detector. \\

The network consisted of a single hidden layer with 12288 units, an input layer of 4096 units, and 256 output units. This enables the network to predict 16 x 16 road prediction patches given 64 x 64 aerial image patches. The network was trained by \ac{SGD} to minimize cross-entropy between training labels and the predicted map patches. Furthermore, unsupervised pre-training was used to initialize the parameters of the network.\\

Experiments have been conducted on two large aerial image datasets, and the network achieved good performance both in terms of precision and recall. A problem identified by \cite{Mnih_roads_high_res_aerial_images}, is that the model is penalized for correct predictions because of noisy labels. Smaller roads or paved areas have often not been marked in the dataset. Additionally, the road labels in the dataset have been generated from road center-line vectors with a fixed width, which results in some roads not being covered by the ground truth. This may lead to a decrease in model performance, since the model is penalized for correctly labelling the roads when minimizing the cross-entropy between predictions and inconsistent labels.\\

The problem with inconsistent labels in the context of aerial images was investigated in \citep{Mnih_aerial_images_noisy}. Two loss functions were proposed to deal with label noise found in aerial imagery. This model resembles the patch-based approach used in \cite{Mnih_roads_high_res_aerial_images}. However, a deep neural network consisting of three hidden layers was used. The first two hidden layers are locally connected layers, while the final hidden layer is fully connected. Unlike convolutional neural networks, there are no parameter sharing involved. \\

The proposed deep neural network performed significantly better in terms of precision and recall, compared to the shallow neural network in \citep{Mnih_roads_high_res_aerial_images}. By utilizing the robust loss functions, performance was further improved.\\

\subsubsection{Larger Datasets}
Learning complex decision boundaries and the variations present in the high-resolution aerial imagery requires a lot of training data. In previous studies, much smaller datasets have been used \citep{Mokhtarzade_road_ann} \citep{Song_road_extraction_svm}. Only eight test images ranging from 1600 to 4000 pixels in width and height were utilized to evaluate automatic road extraction approaches in \citep{Mayer_road_test}. The trend in road extraction and land cover classification literature is to use increasingly larger datasets.\\

A high-resolution aerial dataset used to optimize the randomized forest algorithm in \citep{Kluckner_semantic_height}, contains  155 images and covers an area of about 85 square kilometers. Each of these images are 11500 x 7500 pixels in size and have a \ac{GSD} of 8 centimeter.\\

In both \citep{Mnih_roads_high_res_aerial_images} and \citep{Mnih_aerial_images_noisy}, two large datasets were used to optimize neural networks with many parameters. These datasets covers 500 square kilometers at \ac{GSD} of around 1.20 meters per pixel.\\

The Massachusetts Roads Dataset introduced by \cite{MnihThesis}, consist of 1171 aerial images, each with a height and width of 1500 pixels. The dataset covers an area of over 2600 square kilometers, and contains a variety of regions, such as urban, suburban and rural areas. The \ac{GSD} is 1 meter per pixel.\\


\subsubsection{Feature Representation}
Another trend in road detection is to extract features from larger contexts or pixel neighbourhoods. In addition to increasing the classifiers ability to discriminate between objects sharing similar texture and shape, using larger contexts are necessary for images with a low ground sampling distance.\\

\cite{Mokhtarzade_road_ann} proposed using a shallow neural network for road detection. For this approach, a normalized $3 \times 3$ neighbourhood of pixel values was used as input features to a neural network. The neural network is illustrated in Figure \ref{fig:zoej_neural_network}. \\

\begin{figure}
\begin{center}
\includegraphics[width=.6\columnwidth]{figs/zoej.png}
\caption[shallow neural network]{Pixel neighbourhood and shallow neural network used for road detection by \cite{Mokhtarzade_road_ann} }
\label{fig:zoej_neural_network}
\end{center}
\end{figure}

In \cite{Song_road_extraction_svm}, shape description is generated from segmented images, and certain characteristics descriptive of roads, such as being lengthy and narrow, are used to remove objects that share spectral similarities with roads. Each shape's border length, area, pixel count and approximate radius are used for measuring the shape index and density. Road shapes should have a large shape index value and a small density. The suspected non-road shapes are removed by a threshold operation. \todo{To detailed?}\\

A much larger neighbourhood of pixels were utilized as features in \citep{Mnih_aerial_images_noisy} and \citep{Mnih_roads_high_res_aerial_images}. In this case, $64 \times 64$ pixels with minimal pre-processing formed the input to both  networks. Furthermore, the neural network learns suitable features which enables it to distinguish road pixels from non-road pixels. The large neighbourhood is helpful when resolving ambiguities often found in suburban environments, such as flat grey roof-top and roads. \\

  Combining images with height or elevation information can also increase classifier accuracy. For instance, height information makes grey rooftops easier to distinguish from street areas. The effectiveness of combining images and height maps was demonstrated by \cite{Kluckner_semantic_height} where both color and height cues were integrated as features. The height information significantly improved the performance of the classifier.
  

\subsubsection{Conditional Random Fields}
The smoothness assumption is a strong piece of prior knowledge we have about images. Neighbouring pixels tend to influence each other, and are more likely to belong to the same object or class. \ac{CRF} is a way to explicitly model dependencies between neighbouring pixels, and is often utilized in semantic segmentation tasks to obtain a smoother segmentation result. The goal of semantic segmentation is to split an image into disjoint regions, where each region is associated with a certain class label.\\

In the study by \cite{Kluckner_semantic_height}, an approach for land cover classification given high resolution aerial images was presented. Aerial images are segmented according to five classes: Building, tree, waterbody, green area, and streetlayer. Attributes such as color and edge response is extracted from aerial images, and combined with height information to create an efficient feature representation based on covariance matrices. A randomized forest classifier is trained to learn a conditional probability distribution over the possible class labels given the feature representation. The \ac{CRF} approach was tested in combination with this classifier, and was shown to significantly improve accuracy for semantic segmentation tasks. \\ 

Normally, a classifier predicts each label independently. However, for structured prediction tasks, such as segmentation, contextual information can be useful. The \ac{CRF} approach combines graphical modelling and classification. It involves minimizing the cost of a label assignment for a pixel, as well as the cost of this assignment in relation to the neighbouring pixel assignments. The cost of a label assignment, which is called the unary potential can be estimated from a prediction made by a classifier. The neighbourhood cost is calculated through the pairwise class potentials between the input and it's neighbours.\\

\ac{CRF} is often formulated as a graph with $V$ nodes, where each node represents a pixel, and can be assigned a label $l$ from a discrete set of classes, such as grass, tree, roads. The edges are represented by the set $E$, and models the relationship between neighbouring pixels or nodes. The energy of a label assignment is modelled by:

$$E(y) = \sum\limits_{i\in V} \Psi_i(l_i, x_i) + \sum\limits_{i,j\in E}\Psi_{ij}(l_i, l_j),$$

where $\Psi_i(y_i, x_i)$ is the unary potential modelling the likelihood of pixel x having a label $l$. These estimates can be obtained from a classifier.  $\Psi_{ij}(l_i, l_j)$ is the pairwise potential modelling the coherence of neighbouring pixels. The final label assignment $\hat{y}$, which takes the label assignments of neighbouring pixels into account, is obtained by minimizing the energy:  $\hat{y} =argmin_y E(y)$. Whereas the first term of $E(y)$ prefers the label assignment with the lowest cost, the pairwise potentials prefer pixel neighbourhoods to have similar label assignments. This results in the most probable label assignment $\hat{y}$ given the neighbourhood.\todo{Better but good enough?}\\

\ac{CRF} is also commonly used for semantic segmentation tasks involving general scene understanding. \cite{LeCun_semantic} investigated road scene understanding by utilizing semantic segmentation for imagery found in environments encountered by vehicles. In this approach, a \ac{CNN} is trained to extract features and predict image patches. These class predictions are in turn utilized by the \ac{CRF} as unary potentials.\\

The proposed method was tested on Cambridge-driving Labelled Video Database, which contains high resolution images of roads, signs, pedestrians and other objects found in a driving vehicle environment. The use of \ac{CRF} did improve the overall accuracy, especially for large classes such as roads. For classes with less presence in the dataset the accuracy decreased.\\

Improvements in classification accuracy is further shown empirically by \cite{Schindler_random_field_overview}. Several random field techniques were tested on different aerial image datasets with low ground sampling distance. Enforcing a smoothness prior significantly improved accuracy. \\

An approach similar to \ac{CRF} is proposed by \cite{Mnih_roads_high_res_aerial_images}, where a post-processing step is introduced to improve the predictions produced by the neural network by incorporating knowledge about nearby predictions. This is achieved by training another neural network that predicts 16 by 16 map patches using 64 by 64 patches of predictions. The approach was applied to reduce the amount of gaps and disconnected road present in the baseline predictions. 


 
\subsection{Learning by a Curriculum}
Inspired by how humans learn in an organized fashion, \cite{Bengio_curriculumlearning} presented curriculum learning and investigated how machine learning can benefit from modifying the training regime. In curriculum learning, a learner is gradually presented with harder training samples. In order to do this, an ordering criteria that can identify easy samples must be devised. Experiments showed that a simple multi-stage curriculum reduced convergence time and increased accuracy. \\

A study conducted by \cite{Erhan-unsupervised-pre-training} investigated why supervised learning tasks benefit from unsupervised pre-training. It showed that examples presented early on in training have a disproportionate influence on the outcome of the training procedure. Earlier training can trap the \ac{SGD} in a basin of attraction, which can be hard to escape from.  By using a curriculum strategy the learner can potentially be guided to better areas in parameter space and lead to a better local minima. \\

Curriculum learning shares similarities with boosting algorithms such as AdaBoost. This algorithm trains several weak classifiers by iteratively re-weighting the training set, which gradually puts more emphasis on difficult samples. Unlike boosting, curriculum learning starts off training by considering the easiest examples found in the dataset.\\

Active learning \citep{Cohn_active_learning} is also considered related to curriculum learning. In active learning, the learner participates in selecting samples for training. Contrary to a curriculum strategy, an active learner prefers a strategy of picking examples close to the decision boundary, in order to reduce the number of examples necessary for learning a target concept.\\

Formally, training by a curriculum can be seen as gradually increasing the influence of difficult examples. Let $P(z)$ be the target training distribution, and $W_{i}(z)$ be a weight applied to example $z$ at step $1\leq i\leq N$. The weight $W_{i}(z)$ is reweighted at each step, until $W_{N}(z) = 1$ for all examples.  The training distribution  $Q_{i}(z)$ at step $i$:

$$Q_{i}(z)\propto W_{i}(z)P(z)\forall z$$

At each step, examples are reweighed, which changes the training distribution $Q_{i}(z)$. The weights are first increased on examples considered easy.  
At $i=N$ all weights $W_{1}(z)$ are set to one, and the target training distribution $P(z)$ is recovered. This process should iteratively increase the entropy of the distributions $Q_{t}(z)$. For instance, curriculum learning could be achieved by two steps $N=2$. At step $1$ the influence from harder examples are entirely removed by setting $W_{1}(z) = 0$. Whereas, in step $2$ the target training distribution $P(z)$ is recovered by setting all weights $W_{2}(z) = 1$.\\


An experiment was conducted on the task of shape recognition, where images of geometrical shapes were classified. Two artificially generated datasets consisting of 32x32 grey scale images were constructed. The simpler dataset contained only squares, circles, and equilateral triangles, while the complex dataset was composed of all types of rectangles, circles and triangles. Two neural networks were trained for 256 epochs by \ac{SGD} to classify these shapes. The network trained using a curriculum strategy would start by training only on easier examples found in the simple dataset, and switch to the complex after a certain number of epochs. The baseline network would only train using the complex dataset. The best generalization was obtained by the network using a curriculum, where half of the total epochs was spent on easier examples. \\

Another experiment was conducted on a language modelling task, where a learner predicts the most fitting word that can follow a given sentence. The curriculum strategy in this case was to iteratively grow the allowed vocabulary. Only sentences in the dataset where all words were present in the vocabulary would be included in the training set. The network that utilized curriculum learning performed better than the baseline for this task as well.\\

A considerable challenge with curriculum learning is to define an appropriate curriculum strategy that can work well for a task. In this study the sorting strategies were task specific, and not generally applicable. Furthermore, the tasks presented were fairly simple, and sorting strategies were easy to identify. This may not be the case for datasets containing images where a measure of easiness can be harder to define.\\


Curriculum learning's main challenge is to find a sequence for samples that are meaningful, and can facilitate learning. This requires an ordering criteria that sorts samples based on some measure of "easiness". The criterias presented by \cite{Bengio_curriculumlearning} were problem-specific. To address this issue, \cite{Kumar_self_paced_learning} proposed \ac{SPL}. Instead of a teacher providing the algorithm with a fixed curriculum, the algorithm iteratively selects samples based on its own abilities. \\

\ac{SPL} is an iterative approach, where the learner both selects easy samples and update its parameters at each iteration. The number of samples is determined by a weight that is gradually annealed. At the start of training, only easy samples are considered. In self-paced learning, easy samples are defined as samples that have labels that are easy to predict.
The algorithm finishes when all samples have been considered by the learner, or at convergence.\\

Specifically, \ac{SPL} integrates curriculum learning by modifying the loss function of the model. The model parameters $w$ and binary variables $v_{i}$ are simultaneously estimated by minimizing the modified loss function. The binary variables $v_{i}$, indicate whether the model considers sample $i$ easy or not.  A parameter $K$ is gradually annealed to modify the learning pace by increasing the effect of a regularization term. As $K$ is cooled, harder samples are included in order to minimize the loss. \\

The \ac{SPL} approach was tested on several tasks, including hand-written recognition and object detection. The self-paced learning approach was implemented for a latent structural support vector machine. To predict digits found in the MNIST dataset, self-paced learning performed significantly better on most runs. In the task of object detection, self-paced learning also produced better results.  \\

According to \cite{Lu_self-paced_learning_diversity}, the self-paced learning approach is limited because it does not consider diversity in sample selection. They therefore, proposed an extension to \ac{SPL} called \ac{SPLD}. In this approach, a new regularization term is introduced which encourages selecting diverse samples. \ac{SPLD} was evaluated on different tasks, such as multimedia event detection and video action recognition, and compared against \ac{SPL} and three other baseline methods. In the task of event detection, the \ac{SPLD} method outperformed both \ac{SPL}, RandomForest, and AdaBoost. This was also the case for action recognition.\\

%An unsupervised clustering method is first applied to the samples and the training set. The goal is to identify group of dissimilar samples, and 
%TODO: alternative convex search - fix w fix v.
%\todo{K-means or groups. 


%\todo[inline]{Interesting apporach is to consider noise as a sorting measure for curriculum learning.}

A human behavioral study was conducted by \citep{Khan_human_teach}, in which participants were tasked with teaching a robot a target concept. The goal of the study was to explore what teaching strategies humans employ. Empirical results suggest that human teachers follow the principles of curriculum learning. \\

There are two prominent teaching models in computational teaching. The teaching dimension and the curriculum  learning principle. The former is based on showing samples closest to the decision boundary, in order to minimize the number of samples needed to reveal a target concept. The latter suggests an easy-to-hard teaching strategy. \\

The experiment involved 31 participants, each tasked with teaching a robot the concept of graspability. The robot would not learn anything but followed motions with its gaze. This provided a consistent environment for the trials. Participants were first asked to sort images of common objects based on how easy they are to hold with one hand. The images were placed along a ruler.  The participants then assigned a binary value indicating whether an object is graspable or not to each object. Finally, the participants would act as teachers by showing the robot images and giving a description about the depicted object's graspability. The participants could chose any order, and use as few images that they felt were needed to teach the robot the target concept. The task represents a simple one dimensional machine learning problem of binary classification, where participants assigned x and y value to each sample. \\

Based on the ordering each participant chose, three major human teaching strategies were observed. A large percentage of the participants employed the curriculum learning approach, by gradually presenting samples closer to the decision boundary. 20 of the participants started by showing the most graspable object, and 6 started with the least graspable.  None of the subjects started by showing samples close to the decision boundary, which the teaching dimension model suggests.\\

The experiment showed that there is evidence of curriculum learning being employed by humans in a teacher role. This might indicate that this is an intuitive and efficient way to learn, and might be beneficial for machine learning methods as well. \\


\subsection{Dealing with Noisy Labels}
Supervised learning works well for applications where there are a lot of labelled data available. For object recognition, the large-scale image database ImageNet have often been utilized. This database provide millions of manually annotated and quality controlled images, organized in a semantic hierarchy \citep{Deng_imagenet}. However, for some tasks, such as semantic segmentation and object detection, manually labelled data are expensive and time consuming to create, and high quality datasets can be hard to obtain. Automatically creating large dataset from internet resources, such as image search engines, \ac{GIS} databases, and user annotated images can be very practical in terms of reducing the costs of creating very large datasets. \\

Unfortunately, such datasets will often contain noisy or weak labels. User annotation for images are usually incomplete, image search engines often return images unrelated to the search term, and \ac{GIS} databases might be outdated and missing important object information. This can have negative consequences for a supervised algorithm, which in most cases assume that the labels are correct. These consequences are more evident in datasets containing substantial amount of inconsistent labels.\\
 
There are three main approaches to dealing with noisy label, as outlined in Section \ref{sec:background_theory}. However, in this section only noise-tolerant algorithms are explored, and especially methods that explicitly introduce noise tolerance into deep neural networks.  

%\subsubsection{Modelling the noise distribution}
%Compare the different approaches for modelling the noise.

\subsubsection{Learning to Label Aerial Images from Noisy Data}
The problem with inconsistent labels in aerial images was investigated in \citep{Mnih_aerial_images_noisy}. In this work, two types of label noise were identified,  which datasets constructed from maps are especially susceptible to. Omission noise is defined as objects that appear in the aerial image, but not in the map. Registration noise occurs when the location of the object in the map is inaccurate. Considering that the presence of label noise might negatively impact the classifier accuracy, \cite{Mnih_aerial_images_noisy} proposed two robust loss functions that can be incorporated in a deep learning framework. \\

The first loss function proposed explicitly models asymmetric noise, and is designed to deal with omission noise. It treats label $\tilde{y}$ as a noisy observation generated from true label $y$, according to a noise distribution $p(\tilde{y} \mid y)$. This distribution is determined by two parameters, set before training. The noise model modifies the derivatives produced by the loss function, which results in the neural network being penalized less for making confident, but incorrect predictions. \\

The second loss function is an extension of the first, and considers both omission and registration error. The noise distribution model combines with a generative model, where different crops of an unobserved, perfectly registered map are generated. These crops are used by an expectation–maximization like algorithm to estimate the true label, which can reduce the effect of local registration errors.\\


The loss functions were evaluated on two large aerial road detection datasets. There was a significant improvement in precision and recall for both loss functions, compared to the baseline deep neural network and the network in \citep{Mnih_roads_high_res_aerial_images}. The second loss function performed slightly better than the first for one of the datasets. This dataset had substantial amounts of registration errors.\\


\subsubsection{Training Convolutional Networks with Noisy Labels}
\cite{Sukhbaatar_noisy_network_learning} demonstrate robustness towards label noise in a modified \ac{CNN}. The method models the noise through an additional noise layer which is estimated alongside the network parameters during \ac{SGD} training. The combined model is optimized to predict the noisy labels. The goal of the noise layer is to approximate the noise distribution of the data and thereby forcing the base model to predict the true labels. Experiments were conducted for several datasets, and showed that the approach does well for higher levels of label noise. \\

Like other noise-tolerant methods, the algorithm involves learning a noise distribution, where the examples observed by the algorithm have been altered by a noisy distribution. The noise distribution is parameterized by a matrix Q, where each value specifies the probability of observing noisy label $\tilde{y}$ given the true label $y$: $q_{ji} := p(\tilde{y} = j \mid y = i)$. \\

The matrix Q is implemented by a constrained linear noise layer, which is added to the base model. This is illustrated in Figure \ref{fig:fergus_method}. The weights between the output layer of the base model and the noise layer corresponds to probabilities found in Q. These conditional probabilities $q_{ji}$ are usually unknown, but an approximate noise distribution can be estimated by conventional back-propagation. \\

\begin{figure}
\begin{center}
\includegraphics[width=.6\columnwidth]{figs/Fergusmethod.png}
\caption[Noise matrix Q]{Noise matrix Q is inserted between loss function and output layer of the model. }
\label{fig:fergus_method}
\end{center}
\end{figure}

The training procedure starts with Q fixed to the identify matrix, while the base model is trained. After a number of epochs, the weights in the linear noise layer will also be adapted by back-propagation. To ensure that Q captures the noise properties of the data, a regularization term is used to make it converge to the true noise distribution. Effectively, the prediction of the combined model will be given by:

%Is a regularization term used, or regularizer?
$$p(\tilde{y} = j \mid x) = \sum_{i} q_{ji}p(y = i \mid x),$$ 

where the noisy predictions are made by the conditional probabilities encoded in q and the prediction of the base model. Hopefully, the base model will learn to predict the true labels $y$ instead of the noisy labels $\tilde{y}$.  \\

The approach was tested using the Google street-view house number, CIFAR10 and ImageNet dataset. Noisy labels were synthesized by switching labels of examples with a fixed probability defined by a probability matrix. \\

The noise layer extension consistently achieved better accuracy compared to the baseline network, and also displayed more robustness to inconsistent labelling. Furthermore, the performance of the modified network decreased slower with increasing noise levels. This approach was also efficient in learning the noise distribution. \\

For the ImageNet dataset, the labels were switched on half of the examples in the dataset. The approach did better than the baseline model. Additionally, the approach also outperformed the baseline model trained on the clean unaltered subset of the dataset, showing that the noisy examples carry useful information.


\subsubsection{Training Deep Neural networks on Noisy Labels with Bootstrapping}
In \cite{Reed_noisy_labels_bootstrapping}, a generic approach to handling noisy and incomplete labels in supervised deep learning was presented. The approach incorporate a notion of perceptual consistency in the loss function. A prediction is consistent if the same prediction is made given similar percepts. The learner is allowed to disagree with inconsistent labels by using its own implicit knowledge stored in the network parameters.\\ 

They present two ways of incorporating perceptual consistency in a network. The first involves a reconstruction loss and a noise distribution model. The other method is introduced as bootstrapping, and avoids directly modelling the noise distribution, by using a combination of training labels and the current model's prediction to generate targets.\\

The bootstrapping approach tweaks the loss function to be a convex combination of the model prediction $q$ and the label $y$. The $\beta$ parameter decides the prediction's contribution to the convex combination, and is usually set to a relatively low value. The bootstrapping loss function is denoted:

$$\mathcal{L}(q,y) = - \sum\limits_{k=1}^L [\beta y_k + (1-\beta)z_k]log(q_k),$$

where $z_k$ is assigned a value of 1 for the most probable class $q_i$, given the data. \\

This approach was tested on several image tasks, and yielded substantial improvements for several datasets. For all tasks, deep neural networks were trained and used as the baseline. The networks that were modified to include perceptual consistency were trained by fine-tuning the baseline parameters. \\

The developed method performed better than the baseline. For MNIST handwritten digits dataset, artificially label noise was added. The bootstrapping performed significantly better than the baseline for noise fraction above 35 percent.\\

In the task of emotion recognition using Toronto Faces Database, bootstrapping performed better than the baseline and other approaches. This dataset contains over 4000 face images with emotion labels. This kind of labelling can be subjective and the dataset might therefore contain mislabelled samples. \\

Overall, bootstrapping improves the robustness of a model and is fairly simple to implement. The bootstrapping approach achieves comparable performance to loss functions that require a noise distribution model. Further improvements could be achieved by learning a time-dependent $\beta$ for the loss function.


%-So introducing another softmax layer. Model true class labels oppsed to label observations. Weights between output layer and this new layer should learn the log-probabilities of observing true label j as noisy label k. Doing gradient ascent logP(t|x)  does not do anything yet, because no explicit incentive for te model to treat q as true label. Structure of RMB, t and x conditionally independent given .


%Model trained with a generative objective. Approximate gradient ascent logP(x,t), contrastive divergence. Generative training provides notion of consistency between x and prediction q.

%Feed forward version as follows. Trainable via gradient descent, autoencoder version. :

%$L_{recon}(x,t) = - \sum\limits_{k=1}^L log P(t_k =1 | x) + \beta ||x- Wq(x)||_2^2$

%Multi-class prediction via bostrapping. Consistency objective. Targets convex combination of label and prediction. Cross-entropy loss function. Generate targets for each SGD mini-batch based on current state of the model. Use prediction and labels together to generate targets. Similar to softmax regression with minimum entropy regularization.\\



%Hard boostrapping modifies regression targets using MAP estimates of q given x. So zk is 1 if the q value  considered is the maximum q value

%Used with mini.batch stochastic gradient descent. Leads to EM like algorithm. E step. True confidence targets estimated. M step update model parameter to better predict those generated targets.
%}



\subsubsection{Semi-supervised Learning}
Literature for semi-supervised learning has also covered noisy labels. In semi-supervised learning, a fraction of the dataset is assumed to be correctly labelled or clean, while the remaining data either have no label or is weakly labelled. For tasks where labels are expensive to produce, semi-supervised learning can be advantageous. Furthermore, label noise can have a big impact on semi-supervised learning, since only a small subset of the dataset is labelled. The approaches described below takes an active role in the learning process by iteratively improving the quality of the dataset, which is similar to the bootstrapping technique.\\

Self-training has been suggested as an approach to training a classifier when there is a considerable amount of missing or weak labels present in the dataset \citep{Rosenberg_self-training}. A classifier is trained using an initial set of fully labelled examples, which is then used to predict weakly labelled examples. The set of fully labelled examples is expanded by adding a selection of the predicted examples. The selection can be based on the prediction confidence of the model. The process is repeated, which will incrementally increase the number of fully labelled examples until the entire dataset has been assigned a label. \\

In co-training, proposed by \cite{Blum_co-training}, two classifiers are trained on separate views of the data. This requires two feature sets that are conditionally independent given the class, and that each feature set is sufficient for label predictions. The training set is iteratively expanded by adding unlabelled examples both classifiers can predict with a high confidence. \\
%And agree on prediction, have the same prediction

In \citep{Breve_particle}, particle competition and cooperation is used to address noisy labels in a semi-supervised setting. A graph is constructed from the dataset where each example have a node, and edges connect similar examples. Each labelled example have an associated particle, which will traverse the graph, cooperate with other particles of the same class, and compete with other particle teams. Each particle visits different nodes according to simple rules and will, for each visit, increase the probability of the node belonging to the particle's own class. When the algorithm converges, each node or example is labelled according to the particle team or class that have the largest node probability. The structure of the graph and the particle dynamics will in effect classify unlabelled examples and discover inconsistent labelling. \\

Most of these approaches require an initial dataset that contains clean labels, which, in many cases, require precise manual labelling. For road extraction, a fully labelled, but noisy dataset, can be generated from existing map data. The problem is that we cannot assume that a subset of this dataset contains clean labels, without a thorough inspection. Therefore, techniques that treat the entire dataset as noisy are better suited for this task. 

%The bootstrapping technique share similarities with semi-supervised training. The both rely on their own predictions.


\section{Background discussion}
\label{sec:backgroundDiscussion}
\todo{Breif discussion of what elements influenced thesis project. Modify the text, which were put in future work in preliminary work} The first research question involves reducing the effect of inconsistent labels when training a classifier. The bootstrapping approach presented by \cite{Reed_noisy_labels_bootstrapping} will be evaluated for road detection. Whereas the loss functions presented by \citep{Mnih_aerial_images_noisy} require a noise distribution model,  bootstrapping utilize a  convex combination between the classifier's prediction and the label. Furthermore, the method was tested for several datasets.\\

 An interesting improvement to this approach is to set a time-dependent parameter for how much the classifier's own prediction should be trusted. The parameter should, at the start of training, weight the convex combination more in favor of the label. While in later stages of training, the parameter can favor the classifier's predictions more, because they have become more accurate.\todo{Check if this was done already}\\

It is important to demonstrate that the approach offers robustness to label noise. This can be done by artificially introducing label noise in labels of road detection datasets. Omission noise can be simulated by removing a fraction of the ground truth in the label images, while registration noise can be introduced by shifting subsets of label images by a certain amount of pixels. Classifiers with the bootstrapping extension can be trained on these modified datasets, and compared to the performance achieved from baseline classifiers. The performances can then be compared at different noise levels, and reveal whether or not the technique offers any robustness towards noisy labels. Similar experiments have been conducted by \citep{Sukhbaatar_noisy_network_learning} and \citep{Reed_noisy_labels_bootstrapping} for a variety of datasets. \\

The second research question will investigate what effect a curriculum strategy can have on the road detection system's precision and recall. An interesting curriculum strategy, mentioned by \cite{Bengio_curriculumlearning}, is creating an easy-to-hard ordering of samples according to how noisy the examples are.

To evaluate curriculum learning, the road detection system could be trained, using both an unaltered training scheme and with curriculum learning. The results can then be compared, in order to determine whether curriculum learning can achieve better generalization for road detection.\\

\todo{CRF methods improves results quite a lot , but not a focus of this thesis. Fyll ut}
\todo{Aerial dataset, good dataset to test bootstrapping. Omission and registration errors, since dataset has not been hand-labelled for machine learning purposes. A real dataset, with an application area. }

In summary, the loss function of the road detection system will be modified, and an alternative training regime will be used for training. To measure the effect of these modifications, the system will be compared to a baseline system, and previous results achieved by similar systems.



\chapter{Methods}
\label{cha:architectureAndModel}
\todo[inline]{No code, only figures and maybe pseudocode}
This chapter will describe the architecture,  and the methods needed in order to investigate the research questions presented in Section \ref{sec:Goals and Research Questions}. An overview of the system and it's components can be found in Section \ref{sec:systemOverview}. Section \ref{sec:setup} presents more details about the architecture and the implementation of curriculum learning and bootstrapping loss. The datasets used are presented in Section \ref{sec:datasets}.

\section{System Overview}
\label{sec:systemOverview}
\todo[inline]{Super overview.
Dataset loader (Pre generated or otherwise) Convolutional neural network. Storage. webGUI and in file
}
\todo[inline]{In big terms, talk about finding roads, but as mnih. Datasets contain two label noise. Investigate ways of mitigating the impact label noise have. Also the aerial image datasets contain large variety of images. Some are much harder. Epecially when inconsistent labelling. Curriculum learning. }

\subsection{Data loader}
\todo[inline]{Data loader and network loosely coupled. Can adapt system to other image problems easily. All hyperparameters can be changed in a config file. From number of layers, epochs etc etc. What backpropagation method}

\subsection{Convolutional neural network}
\todo[inline]{Reasons to use CNN. Related work , compelling results. CNN extract features. Suited for image task. }
\todo[inline]{Have to justify model layering. 256 output? why 64x64 input images why? Context. }
\begin{itemize}
\item leaky relu
\item dropout
\item sdg nesterov
\item L2 regularization
\item Large datasets, swap from memory to gpu (maybe)
\item Curriculum stage loading and swapping.
\item Bootstrapping loss
\item Curriculum teacher and dataset example evaluator.
\end{itemize}

The road extraction system was written in Python, and uses the open source library Theano. Theano enables the user to define and evaluate mathematical expressions involving tensors. The library implements several useful features for developing \ac{CNN}s, such as back-propagation, convolution, and max pooling. Training deep neural networks on a \ac{GPU} can be considerably faster than on a \ac{CPU}. Theano can utilize both the \ac{CPU} and \ac{GPU} without making any modifications to the code.\\

The system is based on the deep neural network outlined by \cite{Mnih_aerial_images_noisy}. The network have three convolutional layers and two fully connected layers. This network architecture is depicted in Figure \ref{fig:conv}. After the network is trained, it can predict whether or not roads are present in a $16 \times 16$ pixel area contained in the center of a $64 \times 64$ aerial image patch. The input patch is considerably larger than the output patch, so that the network can better utilize the context in the image. \\

The first layer perform convolution using 16x16 kernels, and outputs in total 64 feature maps. Only the first layer utilize max pooling, which reduces the number of inputs to the next layer as well as introducing some translational invariance to the model. The kernel size in the second and third layer are currently $4 \times 4$ and $3 \times 3$, respectably. The output of the third convolutional layer is used as input to a fully connected neural network with a single hidden layer and an output layer. The latter contains 256 units where each output is the probability of a pixel representing a road.\\

To avoid overfitting the training data, and hopefully achieve better generalization, different regularization schemes are applied during optimization, such as L2 weight decay, early stopping, and the dropout technique. The first applies a weight penalty to prevent weights from growing large. The second stops the optimization process when performance on the validation set starts to consistently decrease. This is an indication of the model starting to overfit the training set. Dropout forces the units to rely less on each other, by randomly disabling half of the units in the network during training. This encourage units to encode independently useful information, since dropout penalize co-adaptation between units.\\

The model parameters are optimized with a special form of \ac{SGD}, called RMSProp. RMSProp keeps a running average of the gradient magnitude for every weight which is used to adaptively adjust the learning rate of each weight. Compared to \ac{SGD}, this will result in a faster convergence. Other optimizers, such as Nesterov momentum, have also been implemented.\\

Before training occurs, a patch dataset suitable for the model is constructed. Each data and label image is rotated by a random amount, and a predefined number of image and label patches are extracted from the images. Before adding the sample to the training set, contrast normalization is applied. The mean pixel value in a patch is subtracted from each pixel, and divided by the standard deviation found for all pixels in the dataset.\\

\subsection{Web Interface}

\section{Experiments/Architcture/hyperparameter}
\label{sec:setup}
\todo[inline]{Bad name}
\subsection{Curriculum learning}
\subsection{Bootstrapping}
\subsection{Network architecture}

\section{Datasets}
\label{sec:datasets}
The experiments will be conducted on two different aerial image datasets. The first is provided by Minh. The performance can be compared to other works. The second dataset have been created from publicly available sources by the author of this thesis. Differences and similarities of these datasets will be further discussed below. \\

\textbf{Massachusetts Roads Dataset}
(Write stats, show example)
\textbf{Massachusetts Buildings Dataset}
\todo[inline]{Include or not?}
\textbf{Norwegian Roads Dataset}
This dataset presents some challenges in terms of label resolution. Road labels registration errors. Fewer omission errors.
Road labels not generated with uniform breadth. Road labels more accurate in terms of road breadth. Images taken mainly from urban and suburban areas throughout norway. In terms of typograhy (Right word?). Cultivated land, mountains, ice, forest areas. Images also range in quality. Different color hues. Vegvesenet --
Ground sampling distance of 0.66. 
The dataset covers an area of approximately 
1200 km2.\\

In addition to the Massachusetts Roads Dataset \citep{MnihThesis}, the proposed algorithm will be tested on a new dataset. This dataset was constructed from aerial images retrieved from Kartverket, which depicts both rural and urban areas in Norway. The labels for this dataset have been generated from road center-line vectors found in the publicly available topographic vector database, N50, provided by Kartverket. \\

Currently, the dataset contains 221 RGB images that are 1536 pixels in width and height. The images have a high resolution with a \ac{GSD} of around 0.66 meters per pixel. There are 184 images in the training set, 21 images in the validation set, and 16 images in the test set. The road center-line vectors are generated as 4 pixel wide black lines in the label images. A cropped image from this dataset can be seen in Figure \ref{fig:aerialimage_norwegian}, with the label image superimposed on the aerial image. Observe that some roads are missing from the label image, as well as the ground truth not covering the roads properly. \\

\begin{figure}[t]
\begin{center}
\includegraphics[width=1\columnwidth]{figs/norwegian_dataset.png}
\caption[Norwegian road dataset example]{Example aerial image with the label image as an overlay.}
\label{fig:aerialimage_norwegian}
\end{center}
\end{figure}

The Norwegian Roads Dataset has been constructed by using QGIS, an open source geographic information system application. The application enables viewing and editing of map data, but also provides a Python interface. A script to create label images was developed, which takes the map coordinates associated with each corner of an aerial image, and generates a raster image of road center-line vectors found inside that area. These raster images can be used as label data in supervised learning. \\

%%\todo[inline]{FROM PRELIMINARY. INCORPORATE ELSEWHERE}
%In this chapter, preliminary work related to the thesis is presented. This includes a convolutional neural network developed for road detection, and a dataset containing aerial images of Norwegian roads, which will be outlined in Section \ref{sec:Method}. In Section \ref{sec:Experiment} some preliminary tests are presented, and future work can be found in Section \ref{sec:Future_work}. 




\chapter{Experiments and Results}
\label{cha:ResearchAndResults}
\todo[inline]{Introduction to experiments and results}

\section{Experimental Design}
\label{sec:experimentalPlan}
To resolve the research questions and test the performance of the road detection system, a number of experiments were planned. Experiments related to the first research question included testing the robustness of the bootstrapping loss function compared to a baseline loss function. For the other research question regarding the performance of curriculum learning, the results from a network trained according to a curriculum strategy was compared with a baseline network. The baseline network was trained using an ordinary dataset with no particular example ordering. In addition, tests measuring the performance of the road detection system were conducted, and the results compared to other works.\\

 Throughout the rest of this chapter the experiments will be referred to by their assigned id. An overview of all experiments and their assigned id, can be found in in Table \ref{tab:planned_experiments}.\\
\begin{table}[htp]
\caption[Experiments overview]{Experiments overview.}
\begin{center}
\begin{adjustbox}{max width=\textwidth}
\begin{tabular}{+l ^l ^l ^p{5cm}}\hline
\rowstyle{\bfseries}
  ID & RQ & Dataset & Description\\\hline
  
  
  E1 & RQ1 & Massachusetts Roads Dataset & Bootstrapping versus baseline at different levels of label noise \\
  E2 & RQ1 & Norwegian Roads Dataset Vbase & Bootstrapping versus baseline at different levels of label noise\\
  E3 & RQ1 & Norwegian Roads Dataset N50/Vbase & Performance of bootstrapping with a label set with large amounts of label noise \\
  E4 & RQ2 & Massachusetts Roads Dataset & Curriculum, baseline and anti-curriculum comparison. \\
  E5 & RQ2 & Norwegian Roads Dataset Vbase & Performance of curriculum learning at different thresholds $D_\theta$. \\
  E6 & RQ2 & Massachusetts Roads Dataset & Curriculum learning with an inexperienced teacher. \\
  E7 & - & Massachusetts Roads Dataset & Best performing road detection network. Larger patch datasets and increased model capacity. \\
  \hline
\end{tabular}
\end{adjustbox}
\end{center}
\label{tab:planned_experiments}
\end{table}

Each experiment found in Table \ref{tab:planned_experiments} has been replicated and measured 10 times. The experiments have many sources of variability, and averaging the measurements improves the reliability of the results. The network, for instance, is initialized with random weights, and the patch dataset is constructed from random sampling. These factors influence the optimization process, and create measurements that can fluctuate.\\

Most of the experiments listed in Table \ref{tab:planned_experiments} compares results from a certain method with a baseline. Running each experiment several times, creates two independent samples from these populations. By statistical hypothesis testing and the Welch's t test, it is possible to assert whether it is plausible that the samples were produced from two distinct populations. Population $i$ can be described by mean $\mu_i$ and variance $\sigma^2_i$. The null hypothesis $H_0\colon \mu_1 = \mu_2$ is rejected in favour of the alternate hypothesis $H_1\colon\mu_1 \neq \mu_2$ , based on the p-value found by the t-test.  If the p-value is below the significance level $\alpha$ of 0.05, $H_0$ is rejected. This indicates that it is unlikely that the samples came from the same underlying population.\\

The Welch's t test is derived from the Student's t test, and is suitable in situations where the variance and mean of the underlying populations are unknown \citep{walpole_probability}. The mean and variance are estimated from the samples. Unlike the Student's t test, the Welch's t test does not assume that the variance of the two populations are equal. However, both tests assume that the populations are normally distributed. This seemed to be true for the experiments in this chapter, based on visual interpretation of normal Q-Q plots created from the measurements. Unfortunately, whether or not the underlying populations were normal could not be confidently asserted because of the small sample size. In appendix -\todo{Create plot appendix}, examples of these plots can be found. \\ 
 
The proposed methods were tested on two different datasets. Despite the datasets involving the same task of road segmentation, they do differ in many ways. The datasets depict separate aerial regions, have different \ac{GSD}, contain different topography and differs in label quality. Conducting experiments on both datasets, might give an indication of whether the methods can be generally applicable or not.\\

For all experiments, most hyperparameters were kept constant. Only the parameters related to the research questions were varied. These hyperparameters are presented in Section \ref{sec:experimentalSetup}. Furthermore, all regularization methods were enabled during testing. The proposed methods should provide some additional benefit when used in combination with a typical configuration of a \ac{CNN}.\\

To show that the bootstrapping loss function is effective at handling inconsistent labelling, it was compared with the cross-entropy loss function at different levels of label noise. To do so, label errors in the form of omission noise, were artificially introduced to the label images. The label degradation involved removing between 0\% and 40\% of the road class pixels from the label images, before sampling the patch datasets. The road removal involved iteratively setting randomly sized regions of label pixels to 0. The experiment should show how well the various loss functions cope with omission noise, and whether bootstrapping loss function is more robust than the cross-entropy loss function. Artificially introducing omission noise was tested with both road datasets in Experiment E1 and E2\\

The bootstrapping loss function was further tested by Experiment E3, where the training set consisted of labels from the label set N50. The more accurate label set Vbase was used by the validation and test set. The N50 label set contains a lot of naturally occurring registration error, which means that the robustness of the different loss functions can be tested without artificial label degradation.\\

The effectiveness of curriculum learning was tested by training the network on two different patch datasets. The first was created according to a curriculum strategy, where each stage $\theta$ only contained examples with a difficulty estimate below $D_\theta$. Subsequent stages increases the difficulty threshold, which results in the network being gradually introduced to harder examples. The second patch dataset was created with no regard to the difficulty, by random sampling. Specifically, it was created with the difficulty threshold $D_\theta =1.0$ for all stages $\theta$. The performance from training the network on this dataset, formed the baseline which was compared to the performance of a network trained on the curriculum dataset. Both datasets contained the same number of stages, and therefore the same number of examples. The network configuration was also identical. Essentially, the only difference was the ordering of the examples in the patch datasets. Curriculum learning was tested with both aerial image datasets in Experiment E4 and E5.\\

The approach of curriculum learning also raises some other interesting questions. How well does a network perform if presented with the "harder" examples first? In Experiment E4, anti-curriculum learning was therefore tested. This resembles the teaching dimension method, as discussed in Section \ref{sec:related_works}, in which harder examples closer to the decision boundary are presented first in order to resolve ambiguities quickly. Furthermore, Experiment E5 explored what happens when setting the difficulty threshold $D_0$ at different values.\\

In Experiment E6, the curriculum strategy was tested with a less experienced curriculum teacher, than in Experiment E4 and E5. This should test whether the proposed curriculum strategy is viable with a teacher that has trained with a limited number of examples. The experiment also tested the effect of just training on the first stage examples. In addition, a more gradual approach of changing the training set distribution was tested. This involved having several smaller stages mixed into the training set by random replacement.\\

In addition to presenting the experimental results as \ac{MSE} test loss per epoch plots, the network performance is also presented as precision and recall curves. This is a common metric for evaluating road detection systems, as discussed in Section \ref{sec:background_theory}.\\

A precision and recall curve is created by thresholding the network's prediction probabilities by values between 0 and 1, and then computing the precision and recall using the binarized predictions and labels. The result is a curve that illustrates the trade-off between precision and recall. As the recall increases, the precision usually decreases. Similar to \citep{Mnih_aerial_images_noisy}, this thesis utilizes a relaxed measure of precision and recall, with a slack variable $p$ set to 3. The relaxed precision is denoted as the fraction of detected road pixels that are within $p$ pixels of a label road pixel. Whereas, the relaxed recall is defined as the fraction of true pixels that are within $p$ pixels of a detected pixel. All experiments listed in Table \ref{tab:planned_experiments} utilized the relaxed version of precision. However, only Experiment E6 and E7 recorded results using the relaxed recall measure, which means that most experiment results have precision and recall values that are a bit more sensitive to minor deviations between predictions and labels.\\

The reason for using the relaxed version of precision and recall is that the majority of the label maps exhibit a small amount of registration noise. It is therefore unreasonable to count predictions which are slightly off target as errors. Consequently, misalignments of 3 pixels or less between the prediction and label will not affect the values of the relaxed precision and recall curve.\\

The results are also presented in tables that include the precision and recall breakeven point, which is derived from the precision and recall curve. This is the point on the precision and recall curve where the precision and recall have an equal value. These points are also marked in the figures by a black dot.\\





\section{Experimental Setup}
\label{sec:experimentalSetup}
The network configuration for most experiments are listed in Table \ref{tab:network_parameters}. Any deviations from this configuration will be detailed in this section. In addition, the specific hyperparameters used for each experiment can be found at \url{http://interface.ml/experiments}. Descriptions about tools used for conducting the experiments can be found in Appendix \ref{app:tools}. The relevant parameters for each experiment are detailed below.\\ 

\subsubsection{E1 - Curriculum Learning with Norwegian Roads Dataset}
This experiment involved comparing two patch datasets generated from the Norwegian Roads Dataset. Both datasets have $N=2$ stages, where each stage includes 110 000 training examples. The first patch dataset was created according to a curriculum strategy, where the difficulty estimate $d(y, q)$ is less than a threshold $D_\theta$. The first stage consists of only examples with a difficulty estimate below 0.25,  while the second stage has a threshold of 1.0, which is equivalent to random sampling of patches. The second patch dataset has a threshold $D_\theta$ of 1.0 for both stages, and constitutes the baseline. Switching between the first and second stage, happened by entirely replacing the training set with examples from the second stage \todo{A bit hard to past tense it!}.\\

The curriculum teacher, which generates predictions for the difficulty estimation are a previous trained model. The teacher classifier was trained with a dataset consisting of 440000 examples for 175 epochs. Otherwise, the network parameters closely resembles the default parameters listed in Table \ref{tab:network_parameters} \todo{True?}. The classifier's final \ac{MSE} test loss was 0.0222, and the relaxed precision and recall breakeven was around 0.71. \\

The default network configuration was used for networks trained on both the curriculum dataset and the baseline dataset. Essentially, the only real difference between the two, is the first stage of the datasets. The models were trained for 120 epochs, with a stage switch at epoch 50. Important parameters for this experiment are listed in Table \ref{tab:key_parameter_E1}.\\

An additional patch dataset has also been included, which tested the performance of anti-curriculum learning. For this dataset, the first stage only include road patches with a difficulty estimate $d(y, q)$ above 0.25.

\begin{table}[h]
\caption[Parameters for Experiment E1]{Key parameters for Experiment E1.}
\begin{center}
\begin{adjustbox}{max width=\textwidth}
\begin{tabular}{+l ^p{11cm}}\hline
\rowstyle{\bfseries}
  - & parameters \\\hline
  Baseline & 120 epochs, s=220000, $d(y, q) < D_{\theta}$, $D_{0} = 1.00$, $D_{1} = 1.0$, $t_{start} = 50$  \\
  Curriculum & 120 epochs, s=220000, $d(y, q) < D_{\theta}$, $D_{0} = 0.25$, $D_{1} = 1.0$, $t_{start} = 50$ \\
  Anti-curriculum & 120 epochs, s=220000, $d(y, q) > D_{\theta}$, $D_{0} = 0.25$, $D_{1} = 0.0$, $t_{start} = 50$ \\\hline
\end{tabular}
\end{adjustbox}
\end{center}
\label{tab:key_parameter_E1}
\end{table}

\subsubsection{E2 - Curriculum Learning with Massachusetts Roads Dataset}
Experiment E2 had a similar setup to Experiment E1. There were two patch datasets, each with two stages, where the first patch dataset has been created by a curriculum strategy and the other by random sampling. In addition to comparing a baseline setup to a curriculum setup, various $D_{0}$ values for stage $0$ were tested. This will show how setting the threshold for the first stage will affect the performance of curriculum learning. Special parameters for this experiment can be found in Table \ref{tab:key_parameter_E2}.\\

\begin{table}[!h]
\caption[Parameters for Experiment E2]{Key parameters for Experiment E2.}
\begin{center}
\begin{adjustbox}{max width=\textwidth}
\begin{tabular}{+l ^p{11cm}}\hline
\rowstyle{\bfseries}
  - & parameters \\\hline
  Baseline & 100 epochs, s=221600, $D_{0} = 1.0$,  $D_{1} = 1.0$, $t_{start} = 50$  \\
  Curriculum 0.15 & 100 epochs, s=221600, $D_{0} = 0.15$, $D_{1} = 1.0$, $t_{start} = 50$ \\
  Curriculum 0.25 & 100 epochs, s=221600, $D_{0} = 0.25$, $D_{1} = 1.0$, $t_{start} = 50$ \\
  Curriculum 0.35 & 100 epochs, s=221600, $D_{0} = 0.35$, $D_{1} = 1.0$, $t_{start} = 50$ \\\hline
\end{tabular}
\end{adjustbox}
\end{center}
\label{tab:key_parameter_E2}
\end{table}

The curriculum teacher for this experiment was trained with a patch dataset of 442800 examples, sampled from Massachusetts Roads Dataset. The teacher model was trained for 272 epoch. The model achieved a \ac{MSE} loss of 0.0225, and a relaxed precision and recall breakeven of 0.80 \todo{Relaxed precision only}.\\



\subsubsection{E3 - Bootstrapping with Massachusetts Roads Dataset}
The bootstrapping loss function was in this experiment compared to the cross-entropy loss function. The loss functions were compared by their performance on patch datasets with several levels of label degradation. This test introduced artificial omission noise, by removing roads from the label images. The network was trained for 140 epochs and with 110800 examples. The learning rate was slightly decreased compared to the default configuration. The bootstrapping loss function's $\beta$ parameter, was set at 1.0, and was incrementally decreased after epoch $M =90$, to $\beta_{min} = 0.9$, at a rate of \todo{Rate of what} The parameters relevant for this experiment are listed in Table \ref{tab:key_parameter_E3}.\\

\begin{table}[!ht]
\caption[Parameters for Experiment E3]{Key parameters for Experiment E3.}
\begin{center}
\begin{adjustbox}{max width=\textwidth}
\begin{tabular}{+l ^p{11cm}}\hline
\rowstyle{\bfseries}
  - & parameters \\\hline
  Baseline & 100 epochs, s=110800, $a=0.0011$, $\mathcal{L}$ = cross-entropy, omission noise levels=0\%, 10\%, 20\%, 30\%, 40\%  \\
  Bootstrapping& 100 epochs, s=110800, $a=0.0011$, $\mathcal{L}$ = bootstrapping, $\beta_{max}$=1.0, $\beta_{min}$=0.9, $M$=60, omission noise levels=0\%, 10\%, 20\%, 30\%, 40\% \\\hline
\end{tabular}
\end{adjustbox}
\end{center}
\label{tab:key_parameter_E3}
\end{table}

\subsubsection{E4 - Bootstrapping with Norwegian Roads Dataset N50/VBase}
In contrast to experiment E3, this experiment tested the effect of bootstrapping for labels having a lot of registration noise. The training set consisted of examples from the Norwegian Roads Dataset N50. The label set N50 has coarser road center-line vectors, which results in a lot of registration error compared to the other aerial image datasets. The experiment optimized models with  $s = 165 000$ patch examples, for 140 epochs. The learning rate $a$ was slightly lower than the default. The bootstrapping parameter $\beta$ , was incrementally decreased from $\beta_{max}=1.0$ at epoch $M=90$, to $\beta_{min}=0.8$. Essentially, the bootstrapping loss functions incorporated it's own predictions starting from epoch 90, and then only at a rate of 0.2. \todo{Rewrite}\todo{Also problematic that decrease factor is not mentioned.} Any improvements should therefore be seen in the \ac{MSE} loss after this epoch. In addition, the confident bootstrapping loss function was also tested in this experiment. A summary of the experiment and key parameters can be found in Figure \ref{tab:key_parameter_E4}.\\

\begin{table}[h]
\caption[Parameters for E4]{Key parameters for E4.}
\begin{center}
\begin{adjustbox}{max width=\textwidth}
\begin{tabular}{+l ^p{10cm}}\hline
\rowstyle{\bfseries}
  - & parameters \\\hline
  Baseline & 140 epochs, s=165000, $a=0.0011$, $\mathcal{L}$ = cross-entropy \\
  Bootstrapping&  140 epochs, s=165000, $a=0.0011$, $\mathcal{L}$ = bootstrapping, $\beta_{max}$=1.0, $\beta_{min}$=0.8, $M$=90\\
    Confident bootstrapping & 140 epochs, $a=0.0011$, s=165000, $\mathcal{L}$ = confident-bootstrapping, $\beta_{max}$=1.0, $\beta_{min}$=0.8, $M$=90\\
  \hline
\end{tabular}
\end{adjustbox}
\end{center}
\label{tab:key_parameter_E4}
\end{table}

\subsubsection{E5 - Bootstrapping with Norwegian Roads Dataset VBase}
Experiment E5 tested the robustness towards label noise for the Norwegian Roads Dataset Vbase. This was done by comparing the performance of networks with different loss functions at several levels of omission noise. The experiment shared a very similar setup to Experiment E3. However, the parameter $\beta$ was decreased slightly more than E3. The experiment configuration is displayed in Table \ref{tab:key_parameter_E5}.\\

\begin{table}[h]
\caption[Parameters for Experiment E5]{Key parameters for Experiment E5.}
\begin{center}
\begin{adjustbox}{max width=\textwidth}
\begin{tabular}{+l ^p{11cm}}\hline
\rowstyle{\bfseries}
  - & parameters \\\hline
  Baseline & 100 epochs, s=110000, $a=0.0011$, $\mathcal{L}$ = cross-entropy, omission noise levels=0\%, 10\%, 20\%, 30\%, 40\%  \\
  Bootstrapping&  100 epochs, s=110000, $a=0.0011$, $\mathcal{L}$ = bootstrapping, $\beta_{max}=1.0$, $\beta_{min}=0.8$, $M=60$, emission noise levels=0\%, 10\%, 20\%, 30\%, 40\% \\
    Confident bootstrapping & 100 epochs, s=110000, $a=0.0011$, $\mathcal{L}$ = confident-bootstrapping, $\beta_{max}=1.0$, $\beta_{min}=0.8$, $M=60$, omission noise levels=0\%, 10\%, 20\%, 30\%, 40\% \\
  \hline
\end{tabular}
\end{adjustbox}
\end{center}
\label{tab:key_parameter_E5}
\end{table}

\subsubsection{E6 - Performance of the Road Detection System}
The performance of the road detection system was tested in Experiment E6. The patch dataset counts 3985200 examples, which have been randomly sampled from the Massachusetts Roads Dataset. The network was trained for 300 epochs, unless early stopping terminated the optimization. In order to train with such a large dataset, the model trained with only a subset of the examples at any given epoch. The training data was switched with another subset every 30th epoch starting from epoch 50 \todo{rewrite}. At any given epoch, the model was optimized by a total of 442800 examples. This is probably not as effective as using the entire dataset throughout training \todo{Check if this is correct}. Any discrepancies from the default network configuration are listed in Table \ref{tab:key_parameter_E6}.
\begin{table}[h]
\caption[Parameters for Experiment E6]{Key parameters for Experiment E6.}
\begin{center}
\begin{adjustbox}{max width=\textwidth}
\begin{tabular}{+l ^p{11cm}}\hline
\rowstyle{\bfseries}
  - & parameters \\\hline
  Baseline & 300 epochs, $s=3 985 200$, $a=0.0015$, $b=128$, $\mathcal{L}$ = cross-entropy, initial\_patience = 400000  \\
  \hline
\end{tabular}
\end{adjustbox}
\end{center}
\label{tab:key_parameter_E6}
\end{table}

\subsubsection{E7 - Curriculum Learning with an Inexperienced Teacher}
While experiment E1 and E2 had teachers that were trained with over 400000 examples, this  
experiment assumed that there are a limited amount of patches available. The teacher for this experiment was therefore only trained with 221600 examples, which is the same number of examples in each patch dataset. The model was trained for 156 epochs, and achieved a test \ac{MSE} of 0.0253 and a relaxed precision and recall of 0.79.\\

For the experiment involving gradually increasing the difficulty of the training set, there were 326800 examples in total, split between $N=5$ stages. The first stage had 110800 examples, whereas the remaining stages had 54000 examples each. For the baseline, the difficulty theshold $D_\theta$ was set to 1 for every stage. The first stage of the curriculum patch dataset, only allowed examples with a difficulty below 0.25. The key parameters for Experiment E7, are displayed in Table \ref{tab:key_parameter_E7}\\

\begin{table}[!h]
\caption[Parameters for Experiment E7]{Key parameters for Experiment E7.}
\begin{center}
\begin{adjustbox}{max width=\textwidth}
\begin{tabular}{+l ^p{10cm}}\hline
\rowstyle{\bfseries}
  - & parameters \\\hline
  Baseline & 120 epochs, s=221600, $D_{0} = 1.0$,  $D_{1} = 1.0$, $t_{start} = 60$\\
  Curriculum & 120 epochs, s=221600, $D_{0} = 0.25$, $D_{1} = 1.0$, $t_{start} = 60$ \\
  Baseline - First stage only & 120 epochs, s=221600, $D_{0} = 1.0$\\
  Curriculum - First stage only & 120 epochs, s=221600, $D_{0} = 0.25$ \\
  Baseline - Gradual & 120 epochs, s=326800, $D_{\theta} = 1.0, \theta \in \{0, 1, 2, 3, 4\}$, $t_{start} = 60$,  $t_{stage} = 15$\\
  Curriculum - Gradual & 120 epochs, s=326800, $D_{0} = 0.25$, $D_{\theta} = 1.0, \theta \in \{1,2,3,4\}$ , $t_{start} = 60$,  $t_{stage} = 15$ \\\hline
\end{tabular}
\end{adjustbox}
\end{center}
\label{tab:key_parameter_E7}
\end{table}


\section{Experimental Results}
\label{sec:experimentalResults}
\todo[inline]{Display results in suitable representation.}
\todo[inline]{Choose what to present}
\begin{itemize}
\item For each experiment
\item model layering. Same. Reasons. Gave good results on their own.
\end{itemize}

\todo[inline]{Or put this in it's own chapter}
\todo[inline]{Avoid drawing grand conclusions. Only what your data can support}
\todo[inline]{Study tables graphs for unusual things that might raise questions with the reader}

The performance of the system has not been compared to \citep{saito_building_and_roads} and \citep{Mnih_aerial_images_noisy}, due to the fact that the precision and recall metrics described in Section \ref{sec:background_theory} have not been implemented yet.\\

Nonetheless, the center image in Figure \ref{fig:result} illustrates the current performance of the system. In this particular test image, the model is able to identify the majority of the roads present, except for a small gravel road on the right side of the image. There are also a lot of prediction errors, such as roads being disconnected, and prediction artefacts in the forest areas. An interesting observation is that the model also correctly predicts small private roads leading up to houses present in the image. Furthermore, the model detects construction roads in the upper left corner. Since these roads are not present in the label image, the model is penalized for these predictions by the cross-entropy loss function.\\


\begin{figure}[t]
\centering
\includegraphics[width=.32\textwidth]{figs/results_data.jpg}\hfill
\includegraphics[width=.32\textwidth]{figs/results_label.jpg}\hfill
\includegraphics[width=.32\textwidth]{figs/label.png}

\caption{Aerial image, model prediction and label image. The aerial image is part of test set in Massachusetts Roads Dataset}
\label{fig:result}
\end{figure}


\begin{figure}
\begin{subfigure}{0.48\textwidth}
\includegraphics[width=\linewidth]{figs/curr50/loss_compare_validation.png}
\caption{Comparison of validation loss.} \label{fig:curr50_loss}
\end{subfigure}
\hspace*{\fill} % separation between the subfigures
\begin{subfigure}{0.48\textwidth}
\includegraphics[width=\linewidth]{figs/curr50/validation_precision_recall_curve.png}
\caption{Precision and recall comparison.} \label{fig:curr50_pr}
\end{subfigure}
\hspace*{\fill} % separation between the subfigures
\caption{Curriculum learning, with switch after 70 epochs. Models trained with 110800 examples. 10 runs, averaged. Baseline and curriculum} \label{fig:curr50}
\end{figure}


\begin{figure}
\begin{subfigure}{0.48\textwidth}
\includegraphics[width=\linewidth]{figs/curr100/validation_loss_curve.png}
\caption{Comparison of validation loss.} \label{fig:curr100_loss}
\end{subfigure}
\hspace*{\fill} % separation between the subfigures
\begin{subfigure}{0.48\textwidth}
\includegraphics[width=\linewidth]{figs/curr100/validation_precision_recall.png}
\caption{Precision and recall comparison.} \label{fig:curr100_pr}
\end{subfigure}
\hspace*{\fill} % separation between the subfigures
\caption{Curriculum learning, with switch after 50 epochs. Models trained with 221600 examples. 10 runs, averaged} \label{fig:curr100}
\end{figure}


\begin{figure}
\begin{subfigure}{0.48\textwidth}
\includegraphics[width=\linewidth]{figs/curr100/curriculum_loss_curves.png}
\caption{Curriculum loss.} \label{fig:curr100_loss2}
\end{subfigure}
\hspace*{\fill} % separation between the subfigures
\begin{subfigure}{0.48\textwidth}
\includegraphics[width=\linewidth]{figs/curr100/baseline_loss_curves.png}
\caption{Baseline loss.} \label{fig:curr100_epochs_baseline2}
\end{subfigure}
\hspace*{\fill} % separation between the subfigures
\caption{Loss for training, validation and test dataset. 221600 examples.} \label{fig:curr100_loss_epochs}
\end{figure}

\begin{figure}
\begin{subfigure}{0.48\textwidth}
\includegraphics[width=\linewidth]{figs/E1/E1-lc-test.png}
\caption{Comparison of test loss} \label{fig:E1_curr_norway_loss}
\end{subfigure}
\hspace*{\fill} % separation between the subfigures
\begin{subfigure}{0.48\textwidth}
\includegraphics[width=\linewidth]{figs/E1/E1-pr-test.png}
\caption{Precision and recall comparisons.} \label{fig:E1_curr_norway_pr}
\end{subfigure}
\hspace*{\fill} % separation between the subfigures
\caption{E1 - Performance of curriculum learning at different thresholds, $D_{0}$ for Norwegian Roads Dataset Vbase} \label{fig:E1_curriculum_norway}
\end{figure}

\begin{figure}
\begin{subfigure}{0.48\textwidth}
\includegraphics[width=\linewidth]{figs/E2/anticurr100_lc_comparison.png}
\caption{Comparison of test loss} \label{fig:E2_curr_mass_loss}
\end{subfigure}
\hspace*{\fill} % separation between the subfigures
\begin{subfigure}{0.48\textwidth}
\includegraphics[width=\linewidth]{figs/E2/anticurr100_pr_comparison.png}
\caption{Precision and recall comparisons.} \label{fig:E2_curr_mass_pr}
\end{subfigure}
\hspace*{\fill} % separation between the subfigures
\caption{E2 - Performance of curriculum learning and anti-curriculum learning for Massachusetts Roads Dataset} \label{fig:E2_curriculum_mass}
\end{figure}




\chapter{Conclusion}
\label{cha:evaluationAndConclusion}
\todo[inline]{Introduction to evaluation and conclusion}

\section{Discussion}
\label{sec:Discussion}
\todo[inline]{Merits and limitations}

\section{Contributions}~\label{cont}
\label{sec:Contributions}
This thesis sought to test approaches for dealing with noisy labels in real-world datasets. This is an compelling inquiry in the field of machine learning, where the trend of using deep neural networks with a huge number of adjustable parameters, requires large training sets to generalize well. There is an abundance of existing data available online, which can be used for learning. Unfortunately, in many cases this data lacks accurate labels for supervised training. To manually label the data can be expensive, and very time consuming in many domains, such as transcribing speech for speech recognition, and tracing ground truth for semantic segmentation. Automatically generating datasets from existing data sources is a quick and economical solution, but can result in datasets with a lot of label noise.\\

The problem of label noise, was therefore tackled in this thesis by testing two different methods. Bootstrapping modifies the loss function, in order to reduce the impact of inconsistent labels. Curriculum learning modifies the training regime by sorting the training set into stages from "easy" to "hard" examples. The sorting mechanism, or curriculum strategy is based on measuring inconsistencies between label and teacher predictions. Coincidentally, the "hard" examples often have inconsistent labelling, and are presented at a later stage of optimization. In effect, inconsistent examples are less frequent concurrence in the first stage of the curriculum dataset.\\
 
The curriculum strategy can most likely be applied to tasks in other domains, since there is not anything intrinsic to images used to estimate example difficulty. However, the effectiveness of this curriculum strategy has not been verified for other domains than road detection in aerial images.\\

The thesis found that bootstrapping did show some robustness towards label noise. The effect was however not significant. Additionally, the base network also performed surprising well for very high rates of omission noise.\\

Curriculum learning demonstrated consistently improved generalization accuracy in the experiments. The improved accuracy, was observed both for the Massachusetts Roads Dataset, as well as the Norwegian Roads Dataset. Changing the example distribution of only the first stage training set, affected the final outcome of the training procedure.\\



\section{Future Work}
\label{sec:futureWork}
\todo[inline]{How to extend yout eotk. Directions that became obvious during work}
\todo[inline]{Possible solutions for limitations in the work conducted}
incorporate conditional random fields, or post processing neural network. Will yield further improvements.\\

Is it worth including every example at later stages of learning. What works better between filtering and curriculum learning. Exclude some examples?\\

How inexperienced can a teacher model be, and still be able to create an effective curriculum.\\

Further cleanup of prediction images, and converting the segmented road areas into road-centerline vectors. \\

Compare SPL methods to the curriculum approach presented in this thesis.\\

Estimate variability of examples, diverse set of examples in the simple dataset.\\

\todo[inline]{Some larger tests but runtime too large for extensive testing with these quantities}




\backmatter


\appendix
\chapter*{Appendices}
\addcontentsline{toc}{chapter}{Appendices}
\renewcommand{\thesection}{\Alph{section}}
\section{System instructions}
\label{app:system_instructions}
In order to run the core system the following dependencies are required:
\begin{itemize}
\item Python 2.7
\item Theano
\item Numpy
\item Unirest
\end{itemize}

The system has only been tested with Ubuntu 14.04, but it should be possible to run on both Windows and Linux as long as the listed dependencies have been installed. Ubuntu is highly recommended because of a more convenient installation process. \\

A Nvidia GPU is also highly recommended for running the system. In most instances a GPU can give considerable speed improvements when training compared to a CPU. This is critical when having to deal with large datasets and models with millions of parameters. In order for Theano to efficiently use your GPU while training, CUDA Toolkit has to be installed. \\

A graphical user interface can also be utilized for monitoring training. Running experiments can be stopped from this user interface, as well as a debugging option which display examples and model predictions. In addition all experiment data is stored as JSON, and can be viewed in the user interface. This includes, a loss graph, precision and recall curve and hyperparameters used while training. \\

The monitoring system can either be run locally, or installed on a server. All communication between the core system and the monitoring system is done by HTTP messaging. To enable monitoring, enable\_gui should be set to true in the core system's config file.

Utilizing this monitoring system requires these dependencies:
\begin{itemize}
\item Node.js
\item MongoDB
\end{itemize}

For installation instructions, see Appendix \ref{app:monitorInstall} and Appendix \ref{app:ubuntuInstall}. Alternatively, installation guides have been included in the README files of the repositories. The URL of these repositories are listed below:

\begin{itemize}
\item https://github.com/olavvatne/CNN
\item https://github.com/olavvatne/ml-monitor
\end{itemize} 


\section{System Installation Guide - Ubuntu}
\label{app:ubuntuInstall}
This guide will help you install all dependencies required for running Theano with a GPU, which should be done in order to run the system. \\
\noindent Install all dependencies:
\begin{lstlisting}[language=bash]
  $ sudo apt-get install -y gcc g++ gfortran build-essential git
   wget linux-image-generic libopenblas-dev python-dev python-pip 
   python-nose python-numpy python-scipy  
\end{lstlisting}
~\\

\noindent Install Theano:
\begin{lstlisting}[language=bash]
  $ sudo pip install --upgrade --no-deps 
  git+git://github.com/Theano/Theano.git
\end{lstlisting}
~\\

\noindent Download Cuda 7 toolkit:
\begin{lstlisting}[language=bash]
  $ sudo wget http://developer.download.nvidia.com/
  compute/cuda/repos/ubuntu1404/x86_64/
  cuda-repo-ubuntu1404_7.0-28_amd64.deb
\end{lstlisting}
~\\

\noindent Depackage Cuda:
\begin{lstlisting}[language=bash]
  $ sudo dpkg -i cuda-repo-ubuntu1404_7.0-28_amd64.deb  
\end{lstlisting}
~\\

\noindent Install the cuda driver:
\begin{lstlisting}[language=bash]
  $ sudo apt-get update
  $ sudo apt-get install -y cuda  
\end{lstlisting}
~\\

\noindent Append path for Cuda nvcc in PATH and add LD\_LIBRARY\_PATH in the .bashrc file (see Table \ref{tab:install_bash_paths}). Then do a reboot:

\FloatBarrier
\begin{table}[!htbp]
\caption[Paths to include]{Paths to include.}
\begin{center}
\begin{adjustbox}{max width=\textwidth}
\begin{tabular}{ l }
  \hline			
  export PATH=(...):/usr/local/cuda/bin  \\
  export LD\_LIBRARY\_PATH=/usr/local/cuda/lib64 \\
  \hline  
\end{tabular}
\end{adjustbox}
\end{center}
\label{tab:install_bash_paths}
\end{table}
\FloatBarrier

\noindent  Create a theano config file as illustrated in Table \ref{tab:install_theano_config_file}. Name this file .theanorc and place it in your home directory:\\

\FloatBarrier
\begin{table}[!htbp]
\caption[Theano config file]{Theano config file.}
\begin{center}
\begin{adjustbox}{max width=\textwidth}
\begin{tabular}{ l }
  \hline			
  ~[global] \\
  floatX=float32 \\
  device=gpu \\
  mode=FAST\_RUN \\
  \\
  ~[nvcc] \\
  fastmath=True \\
  \\
  ~[cuda] \\
  root=/usr/local/cuda \\
  \hline  
\end{tabular}
\end{adjustbox}
\end{center}
\label{tab:install_theano_config_file}
\end{table}
\FloatBarrier

\noindent Clone CNN repository, which contains the core system:
\begin{lstlisting}[language=bash]
  $ git clone https://github.com/olavvatne/CNN.git 
\end{lstlisting}
~\\

\noindent Navigate to the root of the cloned repository. Create a secret.py file, and put the token created for the Monitoring Interface in a variable. See Appendix \ref{app:monitorInstall}:
\begin{lstlisting}[language=bash]
  token = "Bearer " + ml-monitor-token
\end{lstlisting}
~\\

\section{Monitoring Interface Installation Guide - Ubuntu}
\label{app:monitorInstall}
This guide outlines the steps required for setting up the monitoring interface. This guide can also be found in the repository's README file. Some of the steps are slightly different for Windows. \\

\noindent Clone the ml-monitor repository:
\begin{lstlisting}[language=bash]
  $ git clone https://github.com/olavvatne/ml-monitor.git
\end{lstlisting}
~\\

\noindent Install Node.js on your system:
\begin{lstlisting}[language=Python]
  $ sudo apt-get update
  $ sudo apt-get install nodejs
\end{lstlisting}
~\\

\noindent Install npm package manager:
\begin{lstlisting}[language=bash]
    Install npm package manager:
\end{lstlisting}
~\\

\noindent Install MongoDB:
\noindent Navigate to ml-monitor, and install the dependencies of ml-monitor:
\begin{lstlisting}[language=bash]
    $ npm install
\end{lstlisting}
~\\

\noindent Before running ml-monitor, some database collections and a user have to be created in Mongo Shell:
\begin{lstlisting}[language=bash]
    $ export LC_ALL=C (Optional. Might be necessary)
    $ mongo
\end{lstlisting}
~\\

\noindent Inside Mongo Shell, first create a new database:
\begin{lstlisting}[language=bash]
    > use ml-monitor
\end{lstlisting}
~\\

\noindent Then create the database collections required by ml-monitor:
\begin{lstlisting}[language=bash]
    > db.createCollection('experimentlist')
    > db.createCollection('userlist')
    > db.createCollection('grouplist')
\end{lstlisting}
~\\

\noindent Create a user. The authentication system is very simple, so remember to create a long random string of characters as token:
\begin{lstlisting}[language=bash]> 
    db.userlist.insert(
    {"user": "ola", "password": "password", "token": "String"})
\end{lstlisting}
~\\

\noindent Finally, insert the default group, that new experiments will be assigned to:
\begin{lstlisting}[language=bash]
    > db.grouplist.insert(
    {"name": "unassigned", "gid": "0", "date_created": new Date()})
\end{lstlisting}
~\\

\noindent To start the system, run:
\begin{lstlisting}[language=bash]
    $ npm run start
\end{lstlisting}

\section{Experiment tools overview}
\label{app:tools}
All tools which have been created for this thesis are listed below. The source code can be found inside the tools module of the road detection system's repository.
\begin{itemize}
\item measurement.precisionrecall.py\\
The tool creates the precision and recall curve. Command line options can be supplied.
\item layer.visualize.py\\
Opens a params.pkl file, containing the weight and hyperparameter configuration of a trained network, and created a visualization of the kernels from the network's input layer.
\item distribution.curriculum\_diff.py\\
Samples patches and create a histogram showing a patch dataset difficulty estimate distribution. Useful for setting threshold $D_\theta$. Also useful for verifying the content in a curriculum patch dataset stage.
\item distribution.dataset\_std.py\\
Tool for finding an estimate of dataset's standard deviation. Used by contrast normalization in the pre-processing step.
\item distribution.label\_dist\\
Counts the percentage of road label pixels in a dataset.
\item curriculum.dataset\_create.py\\
Tool for pre-generate a curriculum patch dataset. Command line options can be passed, to select a teacher, the thresholds $D_\theta$, the teachers optimal thresholding value, and dataset. The tool also comes with a baseline option which generate a staged patch dataset without curating the content of each stage.
\item convert.alpha.py\\
Converts a RGB dataset to RGBA. Python module PIL quirk, which makes RGBA better when rotating the aerial images. Areas not covered by pixels after a rotation, set to transparent with RGBA. With RGB, these pixels are set to black, which results in a lot of patch examples with no content.
\item figure.average\_compare.py\\
Averages experiment runs, and plots the averaged MSE loss and precision and recall curve from the test dataset. The tool also marks the precision and recall breakeven point in the Figure. The resulting figures are used for comparison purposes in this thesis.
\item figure.average\_loss.py\\
Averages the test, validation and training loss from experiment replicates, and plots the result in a figure. 
\item figure.average\_noise\_levels\\
Average the final MSE loss for each replicate experiment, and precision and recall breakeven. 
The figures created by this tools, show the MSE loss and breakeven, over increasing levels of label noise.
\end{itemize}

\section{Road Detection Systems Results}
In this appendix results from the best performing convolutional neural networks are displayed. The precision and recall breakeven point of the model trained on the Massachusetts Roads Dataset is 0.863. Whereas, the breakeven point for the best model trained on the Norwegian Roads Dataset is 0.765.  \\

\begin{figure}[H]
\begin{subfigure}{0.23\textwidth}
\includegraphics[width=\textwidth]{figs/appendix/img1151.jpg}
\caption{ Image. }
\vspace{0.2cm} % separation vertically between the subfigures
\end{subfigure}
\hspace*{\fill} % separation between the subfigures
\begin{subfigure}{0.23\textwidth}
\includegraphics[width=\textwidth]{figs/appendix/label1151.jpg}
\caption{ Label. }
\vspace{0.2cm} % separation vertically between the subfigures
\end{subfigure}
\hspace*{\fill} % separation between the subfigures
\begin{subfigure}{0.23\textwidth}
\includegraphics[width=\textwidth]{figs/appendix/pred1151.jpg}
\caption{ Prediction. }
\vspace{0.2cm} % separation vertically between the subfigures
\end{subfigure}
\hspace*{\fill} % separation between the subfigures
\begin{subfigure}{0.23\textwidth}
\includegraphics[width=\textwidth]{figs/appendix/hit1151.jpg}
\caption{ Hits. }
\vspace{0.2cm} % separation vertically between the subfigures
\end{subfigure}
\begin{subfigure}{0.23\textwidth}
\includegraphics[width=\textwidth]{figs/appendix/img1160.jpg}
\caption{ Image.}
\vspace{0.2cm} % separation vertically between the subfigures
\end{subfigure}
\hspace*{\fill} % separation between the subfigures
\begin{subfigure}{0.23\textwidth}
\includegraphics[width=\textwidth]{figs/appendix/label1160.jpg}
\caption{Label}
\vspace{0.2cm} % separation vertically between the subfigures
\end{subfigure}
\hspace*{\fill} % separation between the subfigures
\begin{subfigure}{0.23\textwidth}
\includegraphics[width=\textwidth]{figs/appendix/pred1160.jpg}
\caption{Prediction.}
\vspace{0.2cm} % separation vertically between the subfigures
\end{subfigure}
\begin{subfigure}{0.23\textwidth}
\includegraphics[width=\textwidth]{figs/appendix/hit1160.jpg}
\caption{Hits.}
\vspace{0.2cm} % separation vertically between the subfigures
\end{subfigure}
\caption{Road extraction results 1 from the Norwegian Roads Dataset} \label{fig:Norway_app_results}
\end{figure}

\begin{figure}[H]
\begin{subfigure}{0.23\textwidth}
\includegraphics[width=\textwidth]{figs/appendix/img1205.jpg}
\caption{ Image. }
\vspace{0.2cm} % separation vertically between the subfigures
\end{subfigure}
\hspace*{\fill} % separation between the subfigures
\begin{subfigure}{0.23\textwidth}
\includegraphics[width=\textwidth]{figs/appendix/label1205.jpg}
\caption{ Label. }
\vspace{0.2cm} % separation vertically between the subfigures
\end{subfigure}
\hspace*{\fill} % separation between the subfigures
\begin{subfigure}{0.23\textwidth}
\includegraphics[width=\textwidth]{figs/appendix/pred1205.jpg}
\caption{ Prediction. }
\vspace{0.2cm} % separation vertically between the subfigures
\end{subfigure}
\hspace*{\fill} % separation between the subfigures
\begin{subfigure}{0.23\textwidth}
\includegraphics[width=\textwidth]{figs/appendix/hit1205.jpg}
\caption{ Hits. }
\vspace{0.2cm} % separation vertically between the subfigures
\end{subfigure}
\begin{subfigure}{0.23\textwidth}
\includegraphics[width=\textwidth]{figs/appendix/img1217.jpg}
\caption{ Image.}
\vspace{0.2cm} % separation vertically between the subfigures
\end{subfigure}
\hspace*{\fill} % separation between the subfigures
\begin{subfigure}{0.23\textwidth}
\includegraphics[width=\textwidth]{figs/appendix/label1217.jpg}
\caption{Label}
\vspace{0.2cm} % separation vertically between the subfigures
\end{subfigure}
\hspace*{\fill} % separation between the subfigures
\begin{subfigure}{0.23\textwidth}
\includegraphics[width=\textwidth]{figs/appendix/pred1217.jpg}
\caption{Prediction.}
\vspace{0.2cm} % separation vertically between the subfigures
\end{subfigure}
\begin{subfigure}{0.23\textwidth}
\includegraphics[width=\textwidth]{figs/appendix/hit1217.jpg}
\caption{Hits.}
\vspace{0.2cm} % separation vertically between the subfigures
\end{subfigure}
\caption{Road extraction results 2 from the Norwegian Roads Dataset.} \label{fig:Norway_app_results}
\end{figure}


\section{Experiment E5 Results}
The results from Experiment E5, were summarised by plotting the final test loss and precision and breakeven point for increasing levels of omission noise. In this appendix, the test loss figures for every noise rate is shown in Figure \ref{fig:E5_all_lc}, while the precision and recall curves are depicted in Figure \ref{fig:E5_all_pr}. Each plot is the average from 10 separate runs.\\
\label{app:fullE3results}
\begin{figure}[H]
\begin{subfigure}{0.31\textwidth}
\includegraphics[width=\textwidth]{figs/E5/pr_0.png}
\caption{ 0\% } \label{fig:app_E5_0_pr}
\vspace{0.2cm} % separation vertically between the subfigures
\end{subfigure}
\hspace*{\fill} % separation between the subfigures
\begin{subfigure}{0.31\textwidth}
\includegraphics[width=\textwidth]{figs/E5/pr_1.png}
\caption{10\% } \label{fig:app_E5_1_pr}
\vspace{0.2cm} % separation vertically between the subfigures
\end{subfigure}
\hspace*{\fill} % separation between the subfigures
\begin{subfigure}{0.31\textwidth}
\includegraphics[width=\textwidth]{figs/E5/pr_2.png}
\caption{20\% } \label{fig:app_E5_2_pr}
\vspace{0.2cm} % separation vertically between the subfigures
\end{subfigure}
\begin{subfigure}{0.31\textwidth}
\includegraphics[width=\textwidth]{figs/E5/pr_3.png}
\caption{ 30\%} \label{fig:app_E5_3_pr}
\vspace{0.2cm} % separation vertically between the subfigures
\end{subfigure}
\hspace*{\fill} % separation between the subfigures
\begin{subfigure}{0.31\textwidth}
\includegraphics[width=\textwidth]{figs/E5/pr_4.png}
\caption{40\%} \label{fig:app_E5_4_pr}
\vspace{0.2cm} % separation vertically between the subfigures
\end{subfigure}
\caption{E5 - Precision and recall breakeven comparisons for several levels of omission noise.} \label{fig:E5_all_pr}
\end{figure}

\begin{figure}[H]
\begin{subfigure}{0.31\textwidth}
\includegraphics[width=\textwidth]{figs/E5/lc_0.png}
\caption{ 0\% } \label{fig:app_E5_0_lc}
\vspace{0.2cm} % separation vertically between the subfigures
\end{subfigure}
\hspace*{\fill} % separation between the subfigures
\begin{subfigure}{0.31\textwidth}
\includegraphics[width=\textwidth]{figs/E5/lc_1.png}
\caption{10\% } \label{fig:app_E5_1_lc}
\vspace{0.2cm} % separation vertically between the subfigures
\end{subfigure}
\hspace*{\fill} % separation between the subfigures
\begin{subfigure}{0.31\textwidth}
\includegraphics[width=\textwidth]{figs/E5/lc_2.png}
\caption{20\% } \label{fig:app_E5_2_lc}
\vspace{0.2cm} % separation vertically between the subfigures
\end{subfigure}
\begin{subfigure}{0.31\textwidth}
\includegraphics[width=\textwidth]{figs/E5/lc_3.png}
\caption{ 30\%} \label{fig:app_E5_3_lc}
\vspace{0.2cm} % separation vertically between the subfigures
\end{subfigure}
\hspace*{\fill} % separation between the subfigures
\begin{subfigure}{0.31\textwidth}
\includegraphics[width=\textwidth]{figs/E5/lc_4.png}
\caption{40\%} \label{fig:app_E5_4_lc}
\vspace{0.2cm} % separation vertically between the subfigures
\end{subfigure}
\caption{E5 - Test loss comparisons for several levels of omission noise.} \label{fig:E5_all_lc}
\end{figure}


\section{TODOs which apply to entire thesis}
\todo[inline]{Remove this part!}
\begin{itemize}
\item curriculum learning and bootstrapping - capitalize?
\item proof read
\item Road extraction, road detection, semantic segmentation, patch-based. Need clarity, and really good understanding of these terms. Check for misuse of these terms!
\item Read sukhbaatar updated paper
\item Additional thanks in preface?
\item Acronym CNN, is ok. But only defined once? Or once per chapter.
\item Check that installation guides and dependencies are correct - Appendix A, B, C
\item Proof read for incorrect future tense
\item Rest of results.
\item Student t test for results. To see if results are significant, or convincing is the better word.uncertainty for every pair.
\item Footnote for mnih and saito - average of 10 or 1?
\end{itemize}
 
\todo[inline]{field of research, brief motivation for the work, what the research topic is, the research approach(es) applied. contributions}
\todo[inline]{Avoid drawing grand conclusions. Only what your data can support}
\todo[inline]{Study tables graphs for unusual things that might raise questions with the reader}


Chapter 5.
\todo[inline]{Limits and merits. Propertly in discussion summary of results?}
\todo[inline]{Main contributions to the field and how significant}.\\

\addcontentsline{toc}{chapter}{Bibliography}
\bibliography{./bibtex/bibliography}

%\chapter{Appendices}
%\label{cha:appendices}
%\listoftodos
\end{document}
