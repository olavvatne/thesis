\section{Contributions}~\label{cont}
\label{sec:Contributions}


This thesis sought to test approaches for dealing with noisy labels in real-world datasets. This is an compelling inquiry in the field of machine learning, where the trend of using deep neural networks with a huge number of adjustable parameters, requires very large training sets to generalize well. There is an abundance of existing data available online. Unfortunately, in many cases this data lacks accurate labels for supervised training. To manually label the data is expensive, and is very time consuming for many domains, such as transcribing speech for speech recognition, and tracing ground truth for semantic segmentation. Automatically generating datasets from existing data sources is a quick and economical solution, but might result in datasets with a lot of label noise.\\

\todo[inline]{Main contributions to the field and how significant}

Since the curriculum  strategy is based on measuring inconsistencies between label and teacher predictions, it can be applied to any supervised learning task. There is not anything intrinsic to images used in this curriculum strategy. However, the effectiveness of this curriculum strategy has not been verified for other domains in this thesis.\\
