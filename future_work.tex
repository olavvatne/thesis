\section{Future Work}
\label{sec:futureWork}
This section presents future work, such as how to further verify the proposed curriculum strategy and bootstrapping loss function. In addition, this section will suggest further improvements of the road detection system.\\

The most compelling improvement of the road detection system is by incorporating \ac{CRF} or a post processing neural network, as described in Section \ref{sec:related_works}. This was tested by \cite{Kluckner_semantic_height} and \cite{Mnih_aerial_images_noisy}, and yielded accuracy improvements. These methods can also result in a smoother image segmentation. \\

To use the road predictions further in GIS applications, the binary prediction images should be converted into road centerline vectors. To do this, the prediction images have to be combined, and the resulting segmentation image must be cleaned. Methods from computer vision might be applicable for this work. For instance in \citep{Song_road_extraction_svm}, shape descriptions were extracted from the segmentation images, which can be used to measure density and shape index. Based on these values, shapes which have measurements not characteristic of roads can be removed by a threshold operation. In addition, morphological operations, such as thinning or skeletonization can be applied to reduce the segmented road regions down to one pixel thick lines.\\

Another potential improvement of the results is by finding better hyperparameters. For instance, the experiments testing the performance of the road detection system showed that the model capacity was initially constrained. Increasing the number of adjustable weights in the convolutional layers improved the precision and recall breakeven point substantially. Combining dropout with max-norm regularization instead of L2 weight decay could also be interesting, since this configuration achieved better test classification error in \citep{Srivastava_dropout}.\\

The bootstrapping methods did not improve performance significantly. This might be related to ill-suited parameters. Further testing of parameter configurations could be considered. In addition, the bootstrapping method could be tested with increasing levels of registration noise. The confident bootstrapping loss function could also be explored further. An interesting comparison of the bootstrapping methods, is mapping the relationship between a decreasing parameter $\beta$ and their resulting performances. Based on the few experiments conducted, it seems that confident bootstrapping might be less sensitive to decreasing the $\beta_{min}$ parameter.\\

There are also some unresolved questions regarding curriculum learning, and the proposed curriculum strategy. For instance, is it worth including examples with a very high difficulty estimate? How do the data cleansing approaches described in Section \ref{sec:BackgroundAndMotivation}, compare to curriculum learning? The thesis also does not properly determine how experienced a teacher classifier has to be in order to create an effective curriculum dataset. Further tests could illuminate the relationship between the teacher classifier's competency and the effectiveness of the curriculum strategy.  \\

The curriculum learning approach should also be tested on a network trained on a very large dataset. This could alternatively be tested by mapping the impact of curriculum learning for increasing training set sizes. The proposed strategy should also be compared to \ac{SPL} \citep{Kumar_self_paced_learning}, which internalizes the curriculum learning mechanism in its loss function.\\

Furthermore, \cite{Lu_self-paced_learning_diversity} illustrated the need for balancing diversity and easiness in curriculum learning. The \ac{SPL} approach is extended by a preference for both easy and diverse examples. This could potentially be done for the proposed curriculum strategy as well. For instance, unsupervised learning techniques, such as clustering, could organize the image patches into groups based on their similarity. The curriculum dataset can then be constructed with an equal representation of every cluster group. This might allow a reduction of the difficulty threshold $D_0$ without negatively impacting the performance.\\

The current version of the curriculum strategy was effective for two different datasets containing aerial imagery. However, the method should be tested for tasks in other domains as well. This might properly determine whether the proposed curriculum strategy can be generally applicable.\\
 

